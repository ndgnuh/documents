\subsection{Tính tổng chuỗi vô hạn}%
\label{sub:tính_tổng_chuỗi_vô_hạn}

Cho hàm $f(z)$ có hữu hạn cực điểm trên $\mathbb C$. Xét chuỗi $S$ định nghĩa bởi:
\begin{equation}
	\label{eq:infsum_sum}
	S = \sum_{n = -\infty}^{\infty} f(n)
\end{equation}

\begin{theorem}
	\label{theorem:residue_sum}
Giả sử $f(z)$ có các cực điểm $\{z_k\}$ và chuỗi $S$ như trên hội tụ. $f(z)$ thỏa mãn với $M\in \mathbb R,\ M > 0$ và $k \in N, k > 2$ thì $f(z) < M/\abs{z}^k$. Khi đó chuỗi $S$ có thể được đánh giá qua công thức:
	\begin{equation*}
		\sum_{n = -\infty}^{\infty} f(n) = - \sum_{\forall z_k} \Res{\pi \cot(\pi z)f(z)}{z=z_k}
	\end{equation*}
\end{theorem}

\textbf{Chứng minh:} Xét miền $D = [-N-\alpha; N + \alpha]^2$ với $N \in \mathbb N$, gọi $C$ là đường cong giới hạn $D$. Chọn $\alpha$ sao cho $\pi\cos(\pi z)f(z)$ giải tích trên $C$ (chọn được vì $f(z)$ có hữu hạn cực điểm còn $\pi\cos(\pi z)$ chỉ có cực điểm nguyên). Theo công thức định lý \ref{theorem:residue_integral}, ta có:

\begin{figure}[ht]
	\begin{tikzpicture}
		\path[draw=black, fill=gray]
		(-3,3)--(3,3)--(3,-3)--(-3,-3) -- (-3,3);
		\path[draw=black]
		(-5,0) [-latex']--(5,0);
		\path[draw=black]
		(0,-5) [-latex']--(0,5);

		\node[anchor=south west] at (0,3) {$i(N+\alpha)$};
		\node[anchor=north east] at (0,-3) {$-i(N+\alpha)$};
		\node[anchor=south east] at (-3,0) {$-(N+\alpha)$};
		\node[anchor=south west] at (3,0) {$(N+\alpha)$};

		\node at (0,3) {\textbullet};
		\node at (0,-3) {\textbullet};
		\node at (-3,0) {\textbullet};
		\node at (3,0) {\textbullet};

		\node at (1.5,1.5) {$D$};
	\end{tikzpicture}
	\centering
	\caption{Miền $D$ và đường cong $C$}
	\label{fig:fig2}
\end{figure}



\begin{equation}
	\label{eq:residue_sum_proof_01}
	\frac{1}{2\pi i}\int_C \pi \cot(\pi z)f(z) dz = \sum \Res{\pi \cot(\pi z) f(z)}{}
\end{equation}

Hàm $\pi \cot(\pi z)$ có các cực điểm cấp 1 tại $n, n\in\mathbb N$, đồng thời:
\begin{equation}
	\label{eq:residue_sum_proof_02}
	\Res{\pi\cot(\pi z)}{z=n} = 1
\end{equation}

Do đó, công thức (\ref{eq:residue_sum_proof_01}) trở thành:
\begin{align}
	\label{eq:residue_sum_proof_03}
	\frac{1}{2\pi i}\int_C \pi \cot(\pi z)f(z) dz &= \sum_{\forall z_k} \Res{\pi \cot(\pi z) f(z)}{z=z_k} + \sum_{\substack{n = -\infty\\n\notin \{z_k\}}}^{\infty} f(n)
\end{align}

Xét hàm $\cot(\pi z)$, $z = x + iy \in C$, tồn tại số thực $A>0$ sao cho:
\begin{align}
	\label{eq:residue_sum_proof_04}
	\abs{\cot(\pi z)}
	&\le \abs{\frac{e^{i\pi z} + e^{-i\pi z}}{e^{iz} - e^{-iz}}}\nonumber\\
	&\le \frac{\abs{e^{i\pi z}}+\abs{e^{-i\pi z}}}{\abs{e^{i\pi z}}-\abs{e^{-i\pi z}}}\nonumber\\
	&\le \frac{e^{y\pi} + e^{-y\pi}}{e^{y\pi}-e^{-y\pi}}\nonumber\\
	&= \frac{1+e^{-2y\pi}}{1-e^{-2y\pi}}\le A
\end{align}

Theo giả thiết và (\ref{eq:residue_sum_proof_04}), ta có:
\begin{align}
	\abs{\pi\int_C \cot(\pi z) f(z) dz}
	&\le \pi AM \int_C \frac{dz}{\abs{z}^k}\nonumber\\
	&= \pi AM \int_C \frac{dz}{\abs{z}^k}
\end{align}
Vì trên đường cong $C$, $\abs{z} > N$ nên:
\begin{equation}
	\abs{\pi\int_C \cot(\pi z) f(z) dz} \le \frac{8 AM\pi}{N^k}(N+\alpha)
\end{equation}
Vì $k > 1$ nên khi $N \to \infty$, ta có:
\begin{equation}
	\lim_{N\to \infty} \int_C \pi \cot(\pi z) f(z)dz  = 0
\end{equation}
hay:
\begin{equation}
	-\sum_{\forall z_k} \Res{\pi \cot(\pi z) f(z)}{z=z_k} = \sum_{\substack{n = -\infty\\n\notin \{z_k\}}}^{\infty} f(n)
\end{equation}

Vì loại bỏ một số hữu hạn các số hạng ra khỏi tổng vô hạn thì tổng vô hạn không làm thay đổi tổng đó, và vì tập các cực điểm $\{z_k\}$ của $f(z)$ là hữu hạn:
\begin{equation}
	S = \sum_{n=-\infty}^{\infty} f(n) = -\sum_{\forall z_k} \Res{\pi\cot(\pi z)f(z)}{z=z_k} \tag{đpcm}
\end{equation}

\begin{example}
	Tính $\sum_{n=1}^\infty 1/n^2$. Hàm $f(z) = 1/z^2$ có cực điểm tại $z=0$. Ta tính thặng dư của $\pi \cot(\pi z) f(z)$ tại $z=0$:
	\begin{align*}
		\Res{\frac{\pi \cot(\pi z)}{z^2}}{z=0} 
		&= \left.\frac{1}{2!}\cdot\frac{d^2}{dz^2}\left(\frac{\pi \cot(\pi z)}{z^2}\cdot z^3 \right) \right\rvert_{z=0}\\
		&= - \frac{ \pi^{2}}{3}
	\end{align*}
\end{example}

Theo định lý \ref{theorem:residue_sum}, ta có:
\begin{equation*}
	 \sum_{n=-\infty}^{\infty} \frac{1}{n^2}=-\Res{\frac{\pi \cot(\pi z)}{z^2}}{z=0}  = \frac{\pi^2}3
\end{equation*}

Vì hàm $f(n)$ chẵn nên:
\begin{equation*}
	\sum_{n=1}^\infty \frac 1 {n^2} = \frac{1}{2}\sum_{n=-\infty}^{\infty} \frac{1}{n^2} = \frac{\pi^2}{6}
\end{equation*}
