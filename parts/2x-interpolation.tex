\subsection{Nội suy đa thức}
	\theoremstyle{definition}
	\begin{definition}[Đa thức nội suy]
		\textit{
			Cho tập dữ liệu \(\{(z_i, y_i)\}_{i=0}^n \subset \mathbb C^2\), trong đó không có \(z_i\) nào trùng nhau. Một đa thức \(p \in C[z]\) với \(\deg(p) \le n\) thỏa mãn
			\[p(z_i) = y_i\]
			được gọi là đa thức nội suy từ tập dữ liệu đã cho.
		}
	\end{definition}

	\begin{theorem}[Nghiệm duy nhất]
		\textit{Với mỗi tập dữ liệu như trên, tồn tại duy nhất một đa thức nội suy \(p\). (Chỉ thừa nhận không chứng minh)}
	\end{theorem}

	\begin{definition}[Đa thức nội suy Lagrange]
		\textit{Cho tập dữ liệu \({\{(z_i, y_i)\}_{i=0}^n \subset \mathbb{C}^2}\). Đa thức nội suy Lagrange từ tập dữ liệu trên được định nghĩa:
		\begin{equation}
			\label{eq:interpolate00}
			L(z) = \sum_{i=0}^n y_i\cdot l_i(z)
		\end{equation}
		với \(l_i(z)\) là:
		\begin{equation}
			\label{eq:interpolate01}
			l_i(z) = \prod_{\substack{0\le k \le n\\k\ne i}} \frac{z-z_k}{z_i-z_k}
		\end{equation}
		}
	\end{definition}

	Đặt hàm phụ \(w(z)\):
	\begin{equation}
		\label{eq:interpolate02}
		w(z) = (z-z_0)\cdot(z-z_1)\cdot(z-z_2)\cdots(z-z_{n-1})\cdot(z-z_n)
	\end{equation}

	Nhận xét:
	\begin{equation}
		\label{eq:interpolate03}
		w'(z_i) = \left.\frac{w(z)}{z-i}\right\vert_{z=z_i}
	\end{equation}


	Dựa vào (\ref{eq:interpolate02}), (\ref{eq:interpolate03}) viết lại (\ref{eq:interpolate01}):
	\begin{equation}
		\label{eq:interpolate04}
		l_i(z) = \frac{w(z)}{w'(z)(z-z_i)}
	\end{equation}

	Xét hàm \(F(t)\) và \(G(t)\) định nghĩa bởi:
	\begin{align}
		\label{eq:interpolate05}
		F(t) &= \frac{w(z)-w(t)}{(z-t)}\\
		\label{eq:interpolate06}
		G(t) &= w(t)
	\end{align}

	Vì \(\frac{F}{G}(t) \) có các cực điểm đơn giảm tại \(z_i\), ta có:
	\begin{align}
		\Res{\frac{F(t)}{G(t)}}{t=z_i} &= \frac{F(z_i)}{G'(z_i)}\nonumber\\
		\label{eq:interpolate06}
		&= \frac{w(z)}{(z-z_i)w'(z_i)}
	\end{align}

	Dựa vào (\ref{eq:interpolate02}),  (\ref{eq:interpolate03}), (\ref{eq:interpolate04}) và (\ref{eq:interpolate06}), ta có công thức mới cho đa thức nội suy Lagrange:
	\begin{equation}
		\label{eq:interpolate07}
		L(z) = \sum_{i=0}^n y_i\cdot \Res{\frac{F}{G}(z)}{z=z_i}
	\end{equation}

	Gọi \(\gamma\) là đường tròn đủ lớn sao cho \(z_i \in \operatorname{int}( \gamma ),\ i=\overline{0,n}\); \(f\) là hàm đơn trị giải tích trên \(\operatorname{int}(\gamma)\) thỏa mãn \(f(z_i) = y_i\). Dựa vào (\ref{eq:interpolate07}) có dạng khác của đa thức nội suy Lagrange:

	\begin{equation}
		L(z) = \frac{1}{2\pi i} \oint_\gamma \frac{w(z) - w(t)}{(z-t)\cdot w(t)}f(t)dt
	\end{equation}