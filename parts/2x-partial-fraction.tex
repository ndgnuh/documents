
	\subsection{Phân tích lớp hàm hữu tỉ}
	Cho đa thức \(p, q \in \mathbb C[z]\) với \(\deg(p) + 1 \le \deg(q)\). Giả sử \(p\) có các không điểm \(z_1,\cdots, z_n\) có các cấp tương ứng là \(m_1, m_2, \cdots, m_n\). Khi đó hàm \(f(z) = p(z)/q(z)\) có thể được viết
	\begin{equation}
		\frac{p(z)}{q(z)}  = \sum_{i = 0}^n \sum_{j = 1}^{m_i} \frac{a_{i,j}}{(z-z_i)^j}\label{eq:parfrac00}
	\end{equation}
	với các hằng số \(a_{i,j} \in \mathbb K\). Trong bài viết này, kết quả trên sẽ được thừa nhận không chứng minh.

	Xét hàm phụ \(f(z) \cdot (z - z_\alpha)^{\beta-1}\) với \(0\le\alpha\le n\) và \(1\le \beta\le m_\alpha\), ta có:
	\begin{align}
		f(z)\cdot(z-z_\alpha)^{\beta-1} &= \sum_{i=0}^n \sum_{j=1}^{m_i} \frac{a_{i,j}}{(z-z_i)^j}\cdot (z-z_\alpha)^{\beta-1}\\
		&= \cdots + \frac{a_{\alpha,\beta}}{z-z_\alpha} + \cdots \label{eq:parfrac01}
	\end{align}

	Dựa vào công thức (\ref{eq:parfrac01}) ta thấy hệ số \(a_{i,j}\) ở trên có thể được tính theo công thức:
	\begin{equation}
		a_{i,j} = \Res{\left[f(z)\cdot(z-z_i)^{j-1}\right]}{z=z_i}\label{eq:parfrac02}
	\end{equation}

	Thay công thức (\ref{eq:parfrac02}) vào (\ref{eq:parfrac00}) ta được:
	\begin{equation}
		f(z) = \sum_{i=0}^n\sum_{j=1}^{m_i} \frac{1}{(z-z_i)^j}\cdot \Res{\left[f(z)\cdot(z-z_i)^{j-1}\right]}{z=z_i} \label{eq:parfrac03}
	\end{equation}

	Trong trường hợp \(f(z)\) chỉ có cực điểm đơn giản, (\ref{eq:parfrac02}) trở thành:
	\begin{equation}
		f(z) = \sum_{i=0}^n \frac{1}{z-z_i}\cdot \Res{f(z)}{z=z_i}
	\end{equation}

	\textbf{Ví dụ:}
	Phân tích hàm \(f(z) = \frac{1}{z^2(z-i)}\). Các cực điểm của \(f(z)\) bao gồm:
	\[
		\begin{array}{cccc}
			z &=& i & \text{(Cực điểm đơn giản)}\\
			z &=& 0 & \text{(Cực điểm cấp 2)}\\
		\end{array}
	\]

	Tính các thặng dư cần thiết:
	\begin{align*}
		\Res{f(z)}{z=i} &= \left.\frac{1}{[z^2(z-i)]'}\right\vert_{z=i} \\&= -1\\
		\Res{f(z)}{z=0} &= \frac{1}{1!} \left.\frac{d}{dz}\left[f(z)\cdot z^2\right]\right\vert_{z=0}\\&= 1\\
		\Res{f(z)(z-0)}{z=0} &= \left.\frac{1}{[z(z-i)]'}\right\vert_{z=0} \\&= i
	\end{align*}

	Sử dụng công thức (\ref{eq:parfrac03}) ta được:
	\begin{align*}
		f(z) = \frac{-1}{z-i} + \frac{1}{z} + \frac{i}{z^2}
	\end{align*}