\subsection{Biến đổi Laplace ngược}
\begin{definition}
	\label{def:laplace_transform}
	[Biển đổi Laplace]
	\textit{Cho \(f(t)\) là hàm xác định với \(t \in [0; +\infty)\). Biến đổi Laplace của hàm \(f(t)\) được định nghĩa: }
	\begin{equation}
		\mathcal{L}\{f\}(s) = F(s)=\int_0^\infty f(t)e^{-st}dt , \quad s \in \mathbb I
		\label{eq:invlaplace01}
	\end{equation}
	\textit{Khi đó hàm \(f(t)\) được gọi là biến đổi Laplace ngược của hàm \(F(s)\), ký hiệu}
	\begin{equation}
		\label{eq:invlaplace02}
		\mathcal{L}^{-1}\{F\}(t) = f(t)
	\end{equation}
\end{definition}

\begin{theorem}
	[Sự tồn tại của biến đổi Laplace]
	Một hàm \(f(t)\) tồn tại biến đổi Laplace \(F(s)\)khi:
	\begin{equation}
		\label{eq:laplaceexist01}
		(\forall t \ge 0),\ (\exists M, k),\ \abs{f(t)} < Me^{kt}
	\end{equation}
\end{theorem}

Hệ quả:
\begin{align}
	\label{eq:laplaceexist02}
	\abs{F(s)} &= \abs{\int_0^\infty f(t)e^{-st}dt}\nonumber\\
		   &\le \int_0^\infty\abs{f(t)}e^{-st}dt\nonumber\\
		   &\le \int_0^\infty Me^{kt-st}dt\nonumber\\
		   &=\frac{M}{s-k}
\end{align}

\begin{definition}
	\label{def:inv_laplace_transform}
	[Công thức Fourier -- Mellin]
	\textit{Cho hàm \(f(t)\) xác định với \(t>0\), \(f(t)\) có biến đổi Laplace
	\[F(s) = \int_0^\infty f(t)e^{-st}dt\] với \(s = \sigma + i\omega\). Khi đó \(F(s)\) có biến đổi Laplace ngược:}
	\begin{equation}
		f(t)=\frac{1}{2\pi i}\int_{\sigma-i\infty}^{\sigma+i\infty} F(s)e^{st}dt
		\label{eq:invlaplace03}
	\end{equation}
\end{definition}


Từ công thức (\ref{eq:invlaplace03}) ta sẽ đi tìm cách biến đổi Laplace ngược sử dụng lý thuyết thặng dư. Xét hàm $F(s)e^{st}$ với các cực điểm $\{s_k\}$. Gọi $C_1$ là nửa bên trái đường tròn tâm $\sigma$ bán kính $R$, $d$ là đường thẳng \(\operatorname{Re}(s) = \sigma\). Đặt $C = d \cup/C_1$. Chọn $\sigma$ và $R$ sao cho $C$ chứa toàn bộ cực điểm của $F(s)e^{st}$.


\begin{figure}[ht]
	\begin{tikzpicture}
		\node[anchor=south east] at (3,0) {$\sigma$};
		\node at (3,0) {\textbullet};
		\node[anchor=south west] at (0,2) {$jR$};
		\node at (0,2) {\textbullet};
		\node[anchor=north west] at (0,-2) {$-jR$};
		\node at (0,-2){\textbullet};
		\node[anchor=north west] at (1.41,1.41) {\(C\)};

		\path[draw=black, ->]
			(3,2) [-latex']arc[dashed, radius=2, start angle=90, end angle=150];
		\path[draw=black]
			(3,2) arc[dashed, radius=2, start angle=90, end angle=270];
		\path[draw=black]
		(0,-4) [-latex']-- (0,4);
		\path[draw=black]
		(-1,0) [-latex']-- (6,0);
		\path[draw=black]
		(3,2) [dashed]-- (0,2);
		\path[draw=black]
		(3,-2) [dashed]-- (0,-2);
		\path[draw=black]
			(3,-3) [-latex']-- (3,-0.5);
		\path[draw=black]
			(3,-0.5) -- (3,3);
	\end{tikzpicture}
	\centering
	\caption{Đường cong $C$}
	\label{fig:fig2}
\end{figure}



Xét tích phân của \(F(s)e^{st}\) trên đường cong \(C\):
\begin{align}
	\nonumber
	\oint_{C} F(s)e^{st}ds &= \int_{C_1}F(s)e^{st}ds + \int_d F(s)e^{st}ds\\
	\label{eq:invlaplace0456}
				  &= \int_{C_1}F(s)e^{st}ds + \int_{\sigma-iR}^{\sigma+iR} F(s)e^{st}ds\\
	\nonumber
				  &= 2\pi i \sum_{\forall s_k}^{\vphantom{n}}\Res{F(s)e^{st}}{s=s_k}
\end{align}

Trên nửa đường tròn \(C_1\), \(s\) được cho bởi công thức \(s=\sigma+Re^{i\phi},{\ \pi/2\le\phi\le3\pi/2}\). Theo hệ quả (\ref{eq:laplaceexist02}):
\begin{align}
	\abs{F(s)} &< \frac{M}{s-k}\nonumber\\
	\implies \abs{F(\sigma+Re^{i\phi})} &< \frac{M}{\sigma-k+Re^{i\phi}} \to 0 \text{ khi } \nonumber R \to \infty\\
	\implies \lim_{R \to \infty} \abs{F(\sigma+Re^{i\phi})} &= 0 \nonumber
\end{align}
hay:
\begin{equation}
	\label{eq:invlaplace08}
	\implies \forall \ \varepsilon > 0,\ \exists\ R,\ \abs{F(\sigma+Re^{i/phi}} < \varepsilon
\end{equation}

Áp dụng (\ref{eq:invlaplace08}) cho tích phân ở trên, ta có:
\begin{align}
	\abs{\int_{C_1}F(s)e^{st}ds} &\le \abs{\int_{\pi/2}^{3\pi/2}F(\sigma+Re^{i\phi})e^{(\sigma+Re^{i\phi})t}d(\sigma+Re^{i\phi})} \nonumber\\
				     &\le \int_{\pi/2}^{3\pi/2}\abs{F(\sigma+Re^{i\phi})e^{(\sigma+Re^{i\phi})t}iRe^{i\phi}}d\phi\nonumber\\
				     &\le R\varepsilon \int_{\pi/2}^{3\pi/2} \abs{e^{(\sigma+R\cos\phi + Ri\sin\phi)t}}d(\phi) \nonumber\\
				     &= R\varepsilon e^{\sigma t} \int_{\pi/2}^{3\pi/2} e^{Rt\cos\phi}d(\phi) \nonumber\\
				     &= R\varepsilon e^{\sigma t} \int_0^{\pi/2} e^{-Rt\sin\phi}d(\phi) \nonumber
\end{align}

Vì trên đoạn \((0,\pi/2)\), \(\sin(\phi) \le 2\phi/\pi\) nên:
\begin{align}
	\label{eq:invlaplace09}
	\abs{\int_{C_1}F(s)e^{st}ds} &\le R\varepsilon e^{\sigma t} \int_0^{\pi/2} e^{-Rt2\phi/\pi}d(\phi) \nonumber\\
				     &\le \frac{\varepsilon R e^{\sigma t}}{t} \left( 1-e^{-Rt} \right)
\end{align}

Do đó với mọi \(t>0\) ta có:
\begin{align}
	\label{eq:invlaplace10}
	\lim_{s \to \infty} \int_{C_1} F(s)e^{st}ds &= 0\nonumber\\
	\lim_{R \to \infty} \int_{\sigma-iR}^{\sigma+iR} F(s)e^{st}ds &= \int_{\sigma-i\infty}^{\sigma+i\infty} F(s)e^{st}ds\nonumber\\
	\implies \int_{\sigma-i\infty}^{\sigma+i\infty} F(s)e^{st}ds &= 2\pi i \sum^{\vphantom{n}}_{\forall s_k}\Res{F(s)e^{st}}{s=s_k}
\end{align}

Từ (\ref{eq:invlaplace10}) và định nghĩa (\ref{def:inv_laplace_transform}) thu được công thức biến đổi Laplace ngược dựa trên thặng dư của $F(s)e^{st}$:
\begin{equation}
	\label{eq:invlaplace_final}
	f(z) = \sum_{\forall s_k}\Res{F(s)e^{st}}{s=s_k}
\end{equation}

\begin{example}
	Tìm biến đổi Laplace ngược của hàm $F(s) = \frac{1}{s(s+4)}$.\par
	$F(s)$ có các cực điểm đơn giản $s = 0$ và $s = -4$. Dễ thấy cực điểm của $F(s)e^{st}$ cũng chính là cực điểm của $F(s)$. Ta tìm các thặng dư tương ứng:\par
	\begin{align*}
		\Res{F(s)e^st}{s=0} &= \left.\frac{e^{st}}{s+4}\right\rvert_{s=0} = \frac{1}{4}\\
		\Res{F(s)e^st}{s=-4} &= \left.\frac{e^{st}}{s}\right\rvert_{s=-4} = \frac{e^{-4t}}{-4}
	\end{align*}\par
	Dùng công thức (\ref{eq:invlaplace_final}) ta thu được biến đổi Laplace ngược:
	\begin{equation}
		f(t) = \frac 1 4 - \frac{e^{-4t}} 4
	\end{equation}
\end{example}


