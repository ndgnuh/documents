\documentclass[11pt]{book}
\usepackage[a4paper]{geometry}
\usepackage{graphicx}
\usepackage{amsmath, amsfonts, amssymb}
\newgeometry{margin=1in}

\title{Hệ điều hành -- buổi 2}
\author{Nguyễn Đức Hùng}
\begin{document}
\chapter{Giới thiệu}
\section{Tiến trình}
Không gian bộ nhớ của chương trình:
\begin{itemize}
\item Mã lệnh (code)
\item Dữ liệu (data)
\item Stack (bộ nhớ First in first out): là cơ chế giúp cho thực hiện chương trình con một cách tuần tự.
\item Heap: bộ nhớ dùng khi chương trình nạp thêm dữ liệu trong runtime
\item Register: thanh ghi (con trỏ)
\end{itemize}

\section{Shell và system call}
Shell giao tiếp giữa ng dùng và hđh. Shell có các chức năng:
\begin{itemize}
\item Nhận lệnh
\item Phân tích lệnh (vd: đúng cú pháp không, etc\ldots)
\item Thực hiện yêu cầu lệnh
\end{itemize}

\section{Phân loại hệ điều hành}
Phân loại theo phần cứng:
\begin{itemize}
\item Mainframe
\item Máy chủ (server)
\item Đa nhân (multicore)
\item Máy tính cá nhân (PC)
\item Di động (mobile)
\item Nhúng (embed)
\item Thời gian thực (real time): độ trễ nhỏ, \textit{vd: những hệ thống điều khiển máy bay.}
\item Hệ điều hành cảm biến (sensor)/smart card
\end{itemize}
Khác:
\begin{itemize}
\item Mất phí
\item Mã nguồn đóng
\item Mã nguồn mở
\end{itemize}

\chapter{Tiến trình}
Tiến trình -- chương trình đang được thực thi. Tài nguyên được cấp lúc tiến trình khởi chạy, và khi chạy có thể được cấp thêm.
Tiến trình chia thành 2 nhóm chính:
\begin{itemize}
\item Tiến trình hệ điều hành
\item Tiến trình người dùng
\end{itemize}
Việc chia nhóm là này có thể phân biệt qua người dùng (user). Tiến trình có thể gồm 1 or nhiều luồng điều khiển. Hệ điều hành đảm nhận hoạt động của tiến trình:
\begin{itemize}
\item Tạo/xóa
\item Điều phối
\item Cung cấp cơ chế đồng bộ, ngăn xung đột giữa các tiến trình
\end{itemize}
\section{Tiến trình}
Trạng thái hệ thống:
\begin{itemize}
\item Vi xử lý
\item Bộ nhớ
\item Thiết bị ngoại vi
\end{itemize}
Việc thực hiện chương trình làm trạng thái hệ thống thay đổi.
\textbf{Tiến trình là một dãy thay đổi các trạng thái hệ thống}
\paragraph{Các trạng thái tiến trình}
\begin{itemize}
\item Khởi tạo
\item Sẵn sàng
\item Đang thực thi
\item Chờ
\item Kết thúc
\end{itemize}
\section{Process scheduling (điều phối tiến trình)}
Các hàng đợi dành cho tiến trình:
\begin{itemize}
\item Job -- queue: tiến trình hệ thống
\item Ready -- queue: tiến trình sẵn sàng đc thực hiện
\item Device -- queues: tiến trình đang đợi
\end{itemize}
\paragraph{Lựa chọn tiến trình trong hàng đợi}
\begin{itemize}
\item Job scheduler: 
\item CPU Scheduler: 
\end{itemize}

\end{document}
