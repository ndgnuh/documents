\chapter{Giới thiệu}
\section{Tiến trình}
Không gian bộ nhớ của chương trình:
\begin{itemize}
\item Mã lệnh (code)
\item Dữ liệu (data)
\item Stack (bộ nhớ First in first out): là cơ chế giúp cho thực hiện chương trình con một cách tuần tự.
\item Heap: bộ nhớ dùng khi chương trình nạp thêm dữ liệu trong runtime
\item Register: thanh ghi (con trỏ)
\end{itemize}

\section{Shell và system call}
Shell giao tiếp giữa ng dùng và hđh. Shell có các chức năng:
\begin{itemize}
\item Nhận lệnh
\item Phân tích lệnh (vd: đúng cú pháp không, etc\ldots)
\item Thực hiện yêu cầu lệnh
\end{itemize}

\section{Phân loại hệ điều hành}
Phân loại theo phần cứng:
\begin{itemize}
\item Mainframe
\item Máy chủ (server)
\item Đa nhân (multicore)
\item Máy tính cá nhân (PC)
\item Di động (mobile)
\item Nhúng (embed)
\item Thời gian thực (real time): độ trễ nhỏ, \textit{vd: những hệ thống điều khiển máy bay.}
\item Hệ điều hành cảm biến (sensor)/smart card
\end{itemize}
Khác:
\begin{itemize}
\item Mất phí
\item Mã nguồn đóng
\item Mã nguồn mở
\end{itemize}
