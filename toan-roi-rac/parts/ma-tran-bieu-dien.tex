Đồ thị $G$ có thể được biểu diễn bằng một ma trận $A(G) = [a_{ij}]$ với $a_{ij}$ là số cạnh nối nữa $v_i$ và $v_j$. Nếu $G$ là đồ thị có hướng, $a_{ij}$ sẽ là số cung đi từ $v_i$ tới $v_j$.

Nếu cạnh $e$ có đầu mút là đỉnh $v$ thì $e$ gọi là cạnh kề của $v$. Đồ thị G cũng có thể biểu diễn bằng ma trận $M = [m_{ij}]$, trong đó $m_{ij} = 1$ nếu đỉnh $v_i$ có cạnh kề $e_j$. Trong trường hợp $G$ là đồ thị có hướng, $m_{ij} = 1$ nếu $v_i$ là đỉnh ra  và $-1$ nếu $v_i$ là đỉnh vào của $e_j$. Nếu không rơi vào những trường hợp trên thì $m_{ij} = 0$.


\begin{figure}[h]
	\begin{center}
		\begin{minipage}[c]{0.3\textwidth}
			\begin{tikzpicture}[thick]
				\node[vertice, label={[label distance=1mm]120:1}] (v1) at (0,0) {};
				\node[vertice, label={[label distance=1mm]30:2}] (v2) at (3,0) {};
				\node[vertice, label={[label distance=1mm]30:3}] (v3) at (3,3) {};
				\node[vertice, label={[label distance=1mm]120:4}] (v4) at (0,3) {};
				\path[draw=black] (v1) 
					--node[anchor=east]{a} (v4) 
					--node[anchor=west]{b} (v2) 
					--node[anchor=west]{c} (v3);
			\end{tikzpicture}
		\end{minipage}
		\begin{minipage}[c]{0.3\textwidth}
			\begin{align*}
				\left(
					\begin{array}{c|cccc}
						&1&2&3&4\\\hline
						1&0&0&0&1\\
						2&0&0&1&1\\
						3&0&1&0&0\\
						4&1&1&0&0
					\end{array}
				\right)
			\end{align*}
		\end{minipage}
		\begin{minipage}[c]{0.3\textwidth}
			\begin{align*}
				\left(
					\begin{array}{c|ccc}
						 &a&b&c\\\hline
						1&1&0&0\\
						2&0&1&1\\
						3&0&0&1\\
						4&1&1&0
					\end{array}
				\right)
			\end{align*}
		\end{minipage}\par
		\begin{minipage}{0.3\textwidth}\begin{center}$G$\end{center}\end{minipage}
		\begin{minipage}{0.3\textwidth}\begin{center}$A(G)$\end{center}\end{minipage}
		\begin{minipage}{0.3\textwidth}\begin{center}$M(G)$\end{center}\end{minipage}
	\end{center}
	\caption{Biểu diễn đồ thị $G$}
	\label{fig:bieu-dien-do-thi-g}
\end{figure}

\begin{figure}[ht]
	\begin{center}
		\begin{minipage}[c]{0.3\textwidth}
			\begin{tikzpicture}[->, >=stealth, thick]
				\node[vertice, label={[label distance=1mm]120:1}] (v1) at (0,0) {};
				\node[vertice, label={[label distance=1mm]30:2}] (v2) at (3,0) {};
				\node[vertice, label={[label distance=1mm]30:3}] (v3) at (3,3) {};
				\node[vertice, label={[label distance=1mm]120:4}] (v4) at (0,3) {};
				\draw (v1)--node[anchor=east]{a} (v4);
				\draw (v4)--node[anchor=west]{b} (v2);
				\draw (v2)--node[anchor=west]{c} (v3);
			\end{tikzpicture}
		\end{minipage}
		\begin{minipage}[c]{0.3\textwidth}
			\begin{align*}
				\left(
					\begin{array}{c|cccc}
						&1&2&3&4\\\hline
						1&0&0&0&1\\
						2&0&0&1&0\\
						3&0&0&0&0\\
						4&0&1&0&0
					\end{array}
				\right)
			\end{align*}
		\end{minipage}
		\begin{minipage}[c]{0.3\textwidth}
			\begin{align*}
				\left(
					\begin{array}{c|rrr}
						 &a&b&c\\\hline
						1&+1& 0& 0\\
						2& 0&-1&+1\\
						3& 0& 0&-1\\
						4&-1&+1& 0
					\end{array}
				\right)
			\end{align*}
		\end{minipage}\par
		\begin{minipage}{0.3\textwidth}\begin{center}$G$\end{center}\end{minipage}
		\begin{minipage}{0.3\textwidth}\begin{center}$A(G)$\end{center}\end{minipage}
		\begin{minipage}{0.3\textwidth}\begin{center}$M(G)$\end{center}\end{minipage}
	\end{center}
	\caption{Biểu diễn đồ thị $G$ (có hướng)}
	\label{fig:bieu-dien-do-thi-g-co-huong}
\end{figure}



Một đồ thị có thể có nhiều ma trận $A(G)$ hoặc $M(G)$ khác nhau, tùy vào cách đánh số cạnh/đỉnh. Sau đây là một số tính chất:
\begin{itemize}
	\item Nếu $A(G)$ là ma trận đối xứng thì đồ thị $G$ vô hướng. 
	\item Nếu $A(G)$ đối xứng, $a_{ij} \in \{0,1\}$ và $a_{ii} = 0$ thì $G$ là đồ thị đơn giản.
\end{itemize}

