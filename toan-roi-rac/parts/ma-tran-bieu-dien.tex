Đồ thị $G$ có thể được biểu diễn bằng một ma trận $A(G) = [a_{ij}]$ với $a_{ij}$ là số cạnh nối nữa $v_i$ và $v_j$. Nếu $G$ là đồ thị có hướng, $a_{ij}$ sẽ là số cung đi từ $v_i$ tới $v_j$.

Nếu cạnh $e$ có đầu mút là đỉnh $v$ thì $e$ gọi là cạnh kề của $v$. Đồ thị G cũng có thể biểu diễn bằng ma trận $M = [m_{ij}]$, trong đó $m_{ij} = 1$ nếu đỉnh $v_i$ có cạnh kề $e_j$. Trong trường hợp $G$ là đồ thị có hướng, $m_{ij} = 1$ nếu $v_i$ là đỉnh ra  và $-1$ nếu $v_i$ là đỉnh vào của $e_j$. Nếu không rơi vào những trường hợp trên thì $m_{ij} = 0$.


\begin{figure}[h]
	\begin{center}
		\begin{minipage}[c]{0.3\textwidth}
			\begin{tikzpicture}[thick]
				\node[vertice, label={[label distance=1mm]120:1}] (v1) at (0,0) {};
				\node[vertice, label={[label distance=1mm]30:2}] (v2) at (3,0) {};
				\node[vertice, label={[label distance=1mm]30:3}] (v3) at (3,3) {};
				\node[vertice, label={[label distance=1mm]120:4}] (v4) at (0,3) {};
				\path[draw=black] (v1) 
					--node[anchor=east]{a} (v4) 
					--node[anchor=west]{b} (v2) 
					--node[anchor=west]{c} (v3);
			\end{tikzpicture}
		\end{minipage}
		\begin{minipage}[c]{0.3\textwidth}
			\begin{align*}
				\left(
					\begin{array}{c|cccc}
						&1&2&3&4\\\hline
						1&0&0&0&1\\
						2&0&0&1&1\\
						3&0&1&0&0\\
						4&1&1&0&0
					\end{array}
				\right)
			\end{align*}
		\end{minipage}
		\begin{minipage}[c]{0.3\textwidth}
			\begin{align*}
				\left(
					\begin{array}{c|ccc}
						 &a&b&c\\\hline
						1&1&0&0\\
						2&0&1&1\\
						3&0&0&1\\
						4&1&1&0
					\end{array}
				\right)
			\end{align*}
		\end{minipage}\par
		\begin{minipage}{0.3\textwidth}\begin{center}$G$\end{center}\end{minipage}
		\begin{minipage}{0.3\textwidth}\begin{center}$A(G)$\end{center}\end{minipage}
		\begin{minipage}{0.3\textwidth}\begin{center}$M(G)$\end{center}\end{minipage}
	\end{center}
	\caption{Biểu diễn đồ thị $G$}
	\label{fig:bieu-dien-do-thi-g}
\end{figure}

\begin{figure}[ht]
	\begin{center}
		\begin{minipage}[c]{0.3\textwidth}
			\begin{tikzpicture}[->, >=stealth, thick]
				\node[vertice, label={[label distance=1mm]120:1}] (v1) at (0,0) {};
				\node[vertice, label={[label distance=1mm]30:2}] (v2) at (3,0) {};
				\node[vertice, label={[label distance=1mm]30:3}] (v3) at (3,3) {};
				\node[vertice, label={[label distance=1mm]120:4}] (v4) at (0,3) {};
				\draw (v1)--node[anchor=east]{a} (v4);
				\draw (v4)--node[anchor=west]{b} (v2);
				\draw (v2)--node[anchor=west]{c} (v3);
			\end{tikzpicture}
		\end{minipage}
		\begin{minipage}[c]{0.3\textwidth}
			\begin{align*}
				\left(
					\begin{array}{c|cccc}
						&1&2&3&4\\\hline
						1&0&0&0&1\\
						2&0&0&1&0\\
						3&0&0&0&0\\
						4&0&1&0&0
					\end{array}
				\right)
			\end{align*}
		\end{minipage}
		\begin{minipage}[c]{0.3\textwidth}
			\begin{align*}
				\left(
					\begin{array}{c|rrr}
						 &a&b&c\\\hline
						1&+1& 0& 0\\
						2& 0&-1&+1\\
						3& 0& 0&-1\\
						4&-1&+1& 0
					\end{array}
				\right)
			\end{align*}
		\end{minipage}\par
		\begin{minipage}{0.3\textwidth}\begin{center}$G$\end{center}\end{minipage}
		\begin{minipage}{0.3\textwidth}\begin{center}$A(G)$\end{center}\end{minipage}
		\begin{minipage}{0.3\textwidth}\begin{center}$M(G)$\end{center}\end{minipage}
	\end{center}
	\caption{Biểu diễn đồ thị $G$ (có hướng)}
	\label{fig:bieu-dien-do-thi-g-co-huong}
\end{figure}



Một đồ thị có thể có nhiều ma trận $A(G)$ hoặc $M(G)$ khác nhau, tùy vào cách đánh số cạnh/đỉnh. Sau đây là một số tính chất:
\begin{itemize}
	\item Nếu $A(G)$ là ma trận đối xứng thì đồ thị $G$ vô hướng. 
	\item Nếu $A(G)$ đối xứng, $a_{ij} \in \{0,1\}$ và $a_{ii} = 0$ thì $G$ là đồ thị đơn giản.
\end{itemize}

\begin{definition}
	[Đẳng cấu]
	Một đẳng cấu giữa $G$ và $H$ là ánh xạ ${f: V(G) \to V(H)}$ sao cho $\forall\ u,v \in V(G)$ thì $f(u)f(v) \in V(H)$. Nếu tìm được một đẳng cấu từ $G$ tới $H$ và ngược lại, ta nói $G$ và $H$ đẳng cấu nhau.
\end{definition}

Đồ thị $G$ và $H$ được gọi là đẳng cấu $\iff$ có thể áp dụng biến đổi trên hàng của $A(G)$ \textbf{và} áp dụng biến đổi tương tự trên cột của $A(G)$ để thu được $A(H)$.

\begin{figure}[h]
	\begin{center}
		\begin{minipage}[c]{0.4\textwidth}
			\begin{center}
				\begin{tikzpicture}[thick]
					\node[vertice, label={[label distance=1mm]120:1}] (v1) at (0,0) {};
					\node[vertice, label={[label distance=1mm]30:2}] (v2) at (3,0) {};
					\node[vertice, label={[label distance=1mm]30:3}] (v3) at (3,3) {};
					\node[vertice, label={[label distance=1mm]120:4}] (v4) at (0,3) {};
					\path[draw=black] (v1) -- (v4) -- (v2) -- (v3);
				\end{tikzpicture}
			\end{center}
		\end{minipage}
		\begin{minipage}[c]{0.4\textwidth}
			\begin{center}
				\begin{align*}
					\left(
						\begin{array}{c|cccc}
						&1&2&3&4\\\hline
							1&0&0&0&1\\
							2&0&0&1&1\\
							3&0&1&0&0\\
							4&1&1&0&0
						\end{array}
					\right)
				\end{align*}
			\end{center}
		\end{minipage}
		\begin{minipage}{0.4\textwidth}\begin{center}$G$\end{center}\end{minipage}
		\begin{minipage}{0.4\textwidth}\begin{center}$A(G)$\end{center}\end{minipage}
		\begin{minipage}[c]{0.4\textwidth}
			\begin{center}
				\begin{tikzpicture}[thick]
					\node[vertice, label={[label distance=1mm]120:5}] (v5) at (0,0) {};
					\node[vertice, label={[label distance=1mm] 30:7}] (v7) at (3,3) {};
					\node[vertice, label={[label distance=1mm] 30:6}] (v6) at (3,0) {};
					\node[vertice, label={[label distance=1mm]120:8}] (v8) at (0,3) {};
					\path[draw=black] (v8) -- (v5) -- (v6) -- (v7);
				\end{tikzpicture}
			\end{center}
		\end{minipage}
		\begin{minipage}[c]{0.4\textwidth}
			\begin{center}
				\begin{align*}
					\left(
						\begin{array}{c|cccc}
							 &5&6&7&8\\\hline
							5&0&1&0&1\\
							6&1&0&1&0\\
							7&0&1&0&0\\
							8&1&0&0&0
						\end{array}
					\right)
				\end{align*}
			\end{center}
		\end{minipage}
		\begin{minipage}{0.4\textwidth}\begin{center}$H$\end{center}\end{minipage}
		\begin{minipage}{0.4\textwidth}\begin{center}$A(H)$\end{center}\end{minipage}
	\end{center}
	\caption{Đẳng cấu $H$ và $G$}
	\label{fig:dong-cau-h-va-g}
\end{figure}



Trong hình \ref{fig:dong-cau-h-va-g}, đẳng cấu $f$ được xác định: $f(1) = 8$; $f(2) = 6$; $f(3) = 7$; $f(4) = 5$. Từ ma trận $A(G)$ có thể thu được $A(H)$ bằng cách đổi chỗ hai hàng $1,4$ và rồi đổi chỗ hai cột tương ứng.

Đẳng cấu đồ thị là quan hệ tương đương, do đó ta có các lớp tương đương. Một lớp tương đương đẳng cấu đồ thị được biểu diễn bằng một đồ thị \textit{không được gán nhãn}. Dễ thấy đồ thị $G$ và $H$ trong hình \ref{fig:dong-cau-h-va-g} cùng thuộc một lớp tương đương. Ta kí hiệu $G = H$ thay vì $G \cong H$. Tương tự, khi nói $H$ là đồ thị con của $G$, điều này có nghĩa $H$ đẳng cấu với một đồ thị con của $G$, hay $G$ chứa một bản sao của $H$. 

Đẳng cấu bảo toàn quan hệ "kề nhau" giữa các cạnh, do đó nếu muốn chứng minh hai đồ thị không đẳng cấu, ta chỉ cần chỉ ra một đặc tính nào đó liên quan tới đỉnh mà chúng khác nhau (\textit{bậc của các đỉnh, kích cỡ của clique lớn nhất hoặc chu kì nhỏ nhất\ldots}).

Hai đồ thị $G$ và $H$ đẳng cấu $\iff \overline{G} \textnormal{ đẳng cấu } \overline{H}$.

\begin{definition}[tự đẳng cấu]
	Một tự đẳng cấu là một đẳng cấu của đồ thị $G$ với chính nó. $G$ gọi là chuyển tiếp đỉnh $\iff ({\forall\ u,v \in V})\ ({\exists f: f(u) = v})$. Tương tự, $G$ gọi là chuyển tiếp cạnh $\iff ({\forall\ e_1,e_2 \in E})\ ({\exists f: f(e_1) = e_2})$.
\end{definition}


