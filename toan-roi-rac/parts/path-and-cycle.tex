\begin{definition}[đường đi]
	\label{def:duong-di}
	Một \textbf{đường đi} (kí hiệu W) có độ dài $k$ là một dãy các \textbf{đỉnh và cạnh} $v_0, e_1, v_1, e_2,\cdots,e_k,v_k$ với $e_i = (v_{i-1},v_i)$. Độ dài của một đường đi kí hiệu là $l(W)$. Đỉnh $v_0$ gọi là đỉnh đầu và $v_k$ gọi là đỉnh cuối. Nếu không có cạnh nào lặp lại, đường đi gọi là đơn. Nếu không có đỉnh nào lặp lại, đường đi gọi là sơ cấp.
\end{definition}
\begin{definition}[chu trình]
	\label{def:chu-trinh}
	 Một chu trình là một đường đi đơn mà có đỉnh đầu và đỉnh cuối trùng nhau. Nếu chỉ có đỉnh đầu và đỉnh cuối trùng nhau thì chu trình gọi là sơ cấp.
\end{definition}
Có thể thấy chu trình với độ dài 1 là một khuyên.
\begin{proposition}
	Đồ thị $G$ là liên thông $\iff ({\exists uv \in E})({ u\in V_1})({v \in V_2})$ với mọi $V_1,\,V_2 \ne \varnothing$ mà $V = V_1\cup V_2,\, V_1 \cap V_2 = \varnothing$ ($V_1,\,V_2$ là phân hoạch không rỗng của $V$)
	\begin{proof}
		Giả sử $G$ liên thông. Khi đó tồn tại một đường đi sơ cấp từ $u$ tới $v$, trên đường đi đó, sau đỉnh cuối cùng của $V_1$ mà nó đi qua là cạnh nối giữa $V_1$ và $V_2$. Giả sử $G$ không liên không. Chứng minh được chiều xuôi.\\
		Gọi $H$ là một thành phần liên thông của $G$, chọn $V_1 = V(H)$. Khi đó không có cạnh nào của $G$ có một đầu mút thuộc $V_1$ và đầu mút còn lại thuộc $V_2$. Bằng cách đảo vế chứng minh được chiều ngược.
	\end{proof}
\end{proposition}

\begin{theorem}
Một cạnh của $G$ là cạnh cắt $\iff$ cạnh đó không thuộc chu trình nào.	
\begin{proof}
	Gọi $e=(uv)$ là một cạnh trong thành phần liên thông $H$ của $G$. Nếu đồ thị thu được từ $H$ bỏ đi $e$ (kí hiệu $H-e$) vẫn liên thông, thì tồn tại một đường đi sơ cấp từ $u$ tới $v$. Có thể thấy đường đi này nếu gắn thêm cạnh $e$ sẽ tạo thành một chu trình . Vậy nếu $e$ không phải cạnh cắt thì sẽ thuộc một chu trình . \\
	Gọi thành phần liên thông của $G$ chứa $e$ là $H$, $e = (x,y)$ nằm trong chu trình $C$. Lấy bất kì $u,v \in H$. Gọi $P$ là một đường đi bất kì từ $u$ tới $v$. Nếu $P$ không chứa $e$ thì ta có đpcm. Nếu $P$ chứa $e$ thì $P$ đi qua $x$ và $y$. Vì $e$ nằm trong chu trình $C$ nên dù bỏ cạnh $e$ thì vẫn tồn tại một đường đi $P'$ từ $x$ tới $y$, chỉ cần thay $x,e,y$ trong $P$ bằng $P'$ ta lại có một đường đi từ $u \to x \to y \to v$. Do đó nếu bỏ cạnh $e$ thì vẫn tồn tại đường đi từ $u$ tới $v$, suy ra $H-e$ liên thông. Vậy nếu cạnh $e$ thuộc một chu trình thì $e$ không phải cạnh cắt.
\end{proof}
\end{theorem}

\begin{proposition}
	Cho đồ thị $G = (V,E)$, $V=\{v_1, v_2,\ldots,v_n\}$ với $n \ge 3$. Nếu có ít nhất 2 trong các đồ thị con $G-v_1,G-v_2,\ldots,G-v_n$ ($G$ xóa đi đỉnh $v_i$) liên thông thì $G$ liên thông.
	\begin{proof}
		Giả sử $G$ không liên thông, $H_i,\, i = \overline{1,k}$ là các thành phần liên thông của $G$. Nếu xóa một đỉnh khỏi $H_i$ thì không làm thay đổi số thành phần liên thông, trừ khi $H_i = K_1$ (giảm đi còn $k-1$ thành phần liên thông). Nếu 2 trong số $G-v_1,G-v_2,\ldots,G-v_n$ liên thông thì $k = 2$ và $H_1 = H_2 = K_1$, mâu thuẫn với $n \ge 3$.
	\end{proof}
\end{proposition}

\begin{corolarry} Mọi đồ thị $G$ mà chứa ít nhất một cạnh thì có ít nhất hai đỉnh không phải là đỉnh cắt\end{corolarry}
\begin{theorem}
Một đồ thị là một đồ thị hai phía $\iff$ nó không chứa chu trình độ dài lẻ.	
\begin{proof}
	Giả sử $G$ là một đồ thị hai phía, khi đó mọi đường đi trên $G$ đều chuyển qua chuyển lại giữa các đỉnh thuộc $V_1$ và $V_2$. Vì vậy, nếu muốn quay lại điểm bắt đầu thì đường đi buộc phải có độ dài chẵn.\\
	Giả sử $G$ không có chu trình độ dài chẵn. $G$ liên thông. Chọn cố định điểm $u$ thuộc $V(G)$. Với mọi điểm $v \in V(G)$, mọi đường đi từ $u$ tới $v$ đều phải có độ dài cùng chẵn hoặc cùng lẻ, nếu không sẽ tạo ra một chu trình độ dài lẻ (mâu thuẫn). Ta phân hoạch $V(G)$ thành 2 tập $V_1 = \{v\mid$ đường đi từ $u$ tới $v$ độ dài lẻ $\}$ và $V_2 = \{v\mid$ đường đi từ $u$ tới $v$ độ dài chẵn $\}$. Thấy được $V_1$ là một tập độc lập vì một cạnh nối giữa hai đỉnh của $V_1$ sẽ tạo ra một chu trình độ dài lẻ. Ta có điều tương tự với $V_2$. Do đó $G$ là một đồ thị hai phía.
\end{proof}
\end{theorem}


