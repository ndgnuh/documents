\begin{definition}
	[Phép hợp] Đồ thị $G = (V,E)$ gọi là hợp của đồ thị $G_1$ và $G_2 \iff V(G) = V(G_1) \cup V(G_2)$ và $E(G) = E(G_1) \cup E(G_2)$. Ký hiệu $G = G_1\cup G_2$. Nếu $V_1 \cap V_2 = \varnothing$ ta viết $G = G_1 + G_2$. Ta cũng định nghĩa $mG$ là hợp $m$ lần các bản sao của $G$.
\end{definition}
\begin{definition}
	[Phép hội] Đồ thị $G = G_1 \lor G_2 \iff$ $V(G) = V(G_1 + G_2) \land E(G) = E(G_1 +G_2) \cup \left\{e=uv\mid u \in V_1, v \in V_2\right\}$ (Đồ thị $G_1 +G_2$ thêm vào những cạnh nối giữa các đỉnh của chúng).
\end{definition}
\begin{figure}[htpb]
\begin{center}
	\begin{minipage}{0.45\textwidth}
		\begin{center}
		\begin{tikzpicture}[thick]
			\node[vertice, fill=white] (p1) at (0,0) {};
			\node[vertice, fill=white] (p2) at (0,2) {};
			\node[vertice] (c1) at (2,0) {};
			\node[vertice] (c2) at (4,0) {};
			\node[vertice] (c3) at (3,2) {};
			\path[draw=black] (p1)--(p2)(c1)--(c2)--(c3)--(c1);
		\end{tikzpicture}
		\end{center}
	\end{minipage}
	\begin{minipage}{0.45\textwidth}
		\begin{center}
		\begin{tikzpicture}[thick]
			\node[vertice, fill=white] (p1) at (0,0) {};
			\node[vertice, fill=white] (p2) at (0,2) {};
			\node[vertice] (c1) at (2,0) {};
			\node[vertice] (c2) at (4,0) {};
			\node[vertice] (c3) at (3,2) {};
			\path[draw=black] 
				(p2)[dashed]--(c1)(p2)[dashed]--(c2)(p2)[dashed]--(c3)
				(p1)[dashed]--(c1)(p1)[dashed]--(c2)(p1)[dashed]--(c3);
			\path[draw=black]
				(p1)--(p2)(c1)--(c2)--(c3)--(c1);
		\end{tikzpicture}
		\end{center}
	\end{minipage}
\end{center}
\caption{hợp/hội đồ thị $p_2$ và $c_3$}
\label{fig:hop_hoi_do_thi_p2_c3}
\end{figure}



\begin{definition}
	[tích descartes] Đồ thị $G = G_1 \times G_2$ là đồ thị mà $V(G) = \left\{(v_i,u_i)\right\}$ $v_i \in V_1,\, u_i \in V_2$. Có $k$ cạnh nối $(v_i,u_i)$ với $(v_j,u_j) \iff {(v_i = v_j)} \land (u_i $ nối với $u_j$ $k$ lần  trong $G_2)$ hoặc ${(u_i = u_j)} \land (v_i $ nối với $v_j$ $k$ lần  trong $G_1)$.
\end{definition}

\begin{figure}[htpb]
\begin{center}
	\begin{minipage}{0.4\textwidth}
		\begin{center}
		\begin{tikzpicture}[thick]
			\node[label={[label distance=1pt]180:$p_1$}, vertice, fill=white] (p1) at (0,0) {};
			\node[label={[label distance=1pt]180:$p_2$}, vertice, fill=white] (p2) at (0,2) {};
			\node[label={[label distance=1pt]180:$c_1$}, vertice, fill=black] (c1) at (2,0) {};
			\node[label={[label distance=1pt]000:$c_2$}, vertice, fill=black] (c2) at (4,0) {};
			\node[label={[label distance=1pt]000:$c_3$}, vertice, fill=black] (c3) at (3,2) {};
			\path[draw=black] (p1)--(p2)(c1)--(c2)--(c3)--(c1);
		\end{tikzpicture}
		\end{center}
	\end{minipage}
	\begin{minipage}[m]{0.05\textwidth}
		$\implies$
	\end{minipage}
	\begin{minipage}{0.45\textwidth}
		\begin{center}
		\begin{tikzpicture}[thick]
			\node[label={[label distance=1pt]270:$(c_1,p_1)$}, vertice] (c1p1) at (0,0) {};
			\node[label={[label distance=1pt]090:$(c_1,p_2)$}, vertice] (c1p2) at (0,2) {};
			\node[label={[label distance=1pt]270:$(c_2,p_1)$}, vertice] (c2p1) at (2,0) {};
			\node[label={[label distance=1pt]090:$(c_2,p_2)$}, vertice] (c2p2) at (2,2) {};
			\node[label={[label distance=1pt]270:$(c_3,p_1)$}, vertice] (c3p1) at (4,0) {};
			\node[label={[label distance=1pt]090:$(c_3,p_2)$}, vertice] (c3p2) at (4,2) {};
			\path[draw=black] 
				(c1p1)--(c2p1)--(c3p1) to[in =  30,out=150] (c1p1)
				(c1p2)--(c2p2)--(c3p2) to[out=-150,in =-30] (c1p2)
				(c3p1)--(c3p2)
				(c2p1)--(c2p2)
				(c1p1)--(c1p2);
		\end{tikzpicture}
		\end{center}
	\end{minipage}
\end{center}
\caption{Tích descartes $P_2 \times C_3$}
\label{fig:tich_descartes_p2_x_c3}
\end{figure}
\begin{definition}
	[tích tensor] Tích tensor của $G$, kí hiệu $G_1 \cdot G_2$, là một đồ thị với $V(G) = V(G_1) \times V(G_2)$. Với $u_i \in V_1$, $v_i \in V_2$, có $k$ cạnh nối giữa $(u_i,v_i)$ và $(u_j,v_j) \iff ($có $m$ cạnh nối $u_i$ với $u_j$, $n$ cạnh nối $v_i$ với $v_j) \land (k = m\cdot n)$.
\end{definition}

\begin{figure}[htpb]
\begin{center}
	\begin{minipage}{0.4\textwidth}
		\begin{center}
		\begin{tikzpicture}[thick]
			\node[label={[label distance=1pt]180:$p_1$}, vertice, fill=white] (p1) at (0,0) {};
			\node[label={[label distance=1pt]180:$p_2$}, vertice, fill=white] (p2) at (0,2) {};
			\node[label={[label distance=1pt]180:$c_1$}, vertice, fill=black] (c1) at (2,0) {};
			\node[label={[label distance=1pt]000:$c_2$}, vertice, fill=black] (c2) at (4,0) {};
			\node[label={[label distance=1pt]000:$c_3$}, vertice, fill=black] (c3) at (3,2) {};
			\path[draw=black] (p1)--(p2)(c1)--(c2)--(c3)--(c1);
		\end{tikzpicture}
		\end{center}
	\end{minipage}
	\begin{minipage}[m]{0.05\textwidth}
		$\implies$
	\end{minipage}
	\begin{minipage}{0.45\textwidth}
		\begin{center}
		\begin{tikzpicture}[thick]
			\node[label={[label distance=1pt]270:$(c_1,p_1)$}, vertice] (c1p1) at (0,0) {};
			\node[label={[label distance=1pt]090:$(c_1,p_2)$}, vertice] (c1p2) at (0,2) {};
			\node[label={[label distance=1pt]270:$(c_2,p_1)$}, vertice] (c2p1) at (2,0) {};
			\node[label={[label distance=1pt]090:$(c_2,p_2)$}, vertice] (c2p2) at (2,2) {};
			\node[label={[label distance=1pt]270:$(c_3,p_1)$}, vertice] (c3p1) at (4,0) {};
			\node[label={[label distance=1pt]090:$(c_3,p_2)$}, vertice] (c3p2) at (4,2) {};
			\path[draw=black]
				(c1p2)--(c2p1)--(c3p2)--(c1p1)--(c2p2)--(c3p1)--(c1p2);
		\end{tikzpicture}
		\end{center}
	\end{minipage}
\end{center}
\caption{Tích tensor $P_2 \cdot C_3$}
\label{fig:tich_tensor_p2_c3}
\end{figure}

\begin{definition}
	[tích strong] Tích strong của $G$, kí hiệu $G = G_1 \circledast G_2$, được định nghĩa $G_1 \circledast G_2 = (G_1 \times G_2) \cup (G_1 \cdot G_2)$.
\end{definition}

\begin{figure}[htpb]
\begin{center}
	\begin{minipage}{0.4\textwidth}
		\begin{center}
		\begin{tikzpicture}[thick]
			\node[label={[label distance=1pt]180:$p_1$}, vertice, fill=white] (p1) at (0,0) {};
			\node[label={[label distance=1pt]180:$p_2$}, vertice, fill=white] (p2) at (0,2) {};
			\node[label={[label distance=1pt]180:$c_1$}, vertice, fill=black] (c1) at (2,0) {};
			\node[label={[label distance=1pt]000:$c_2$}, vertice, fill=black] (c2) at (4,0) {};
			\node[label={[label distance=1pt]000:$c_3$}, vertice, fill=black] (c3) at (3,2) {};
			\path[draw=black] (p1)--(p2)(c1)--(c2)--(c3)--(c1);
		\end{tikzpicture}
		\end{center}
	\end{minipage}
	\begin{minipage}[m]{0.05\textwidth}
		$\implies$
	\end{minipage}
	\begin{minipage}{0.45\textwidth}
		\begin{center}
		\begin{tikzpicture}[thick]
			\node[label={[label distance=1pt]270:$(c_1,p_1)$}, vertice] (c1p1) at (0,0) {};
			\node[label={[label distance=1pt]090:$(c_1,p_2)$}, vertice] (c1p2) at (0,2) {};
			\node[label={[label distance=1pt]270:$(c_2,p_1)$}, vertice] (c2p1) at (2,0) {};
			\node[label={[label distance=1pt]090:$(c_2,p_2)$}, vertice] (c2p2) at (2,2) {};
			\node[label={[label distance=1pt]270:$(c_3,p_1)$}, vertice] (c3p1) at (4,0) {};
			\node[label={[label distance=1pt]090:$(c_3,p_2)$}, vertice] (c3p2) at (4,2) {};
			\path[draw=black]
				(c1p2)--(c2p1)--(c3p2)--(c1p1)--(c2p2)--(c3p1)--(c1p2);
			\path[draw=black] 
				(c1p1)--(c2p1)--(c3p1) to[in =  20,out=160] (c1p1)
				(c1p2)--(c2p2)--(c3p2) to[out=-160,in =-20] (c1p2)
				(c3p1)--(c3p2)
				(c2p1)--(c2p2)
				(c1p1)--(c1p2);
		\end{tikzpicture}
		\end{center}
	\end{minipage}
\end{center}
\caption{Tích strong $P_2 \circledast C_3$}
\label{fig:tich_strong_p2_c3}
\end{figure}
%
%

