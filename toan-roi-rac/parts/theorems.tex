\begin{theorem}
	\label{theo:tong_bac_cua_dinh} Cho một đồ thị $G$, tổng bậc của các đỉnh trong $G$ bằng 2 lần số cạnh của $G$.
	\begin{equation*}
		\sum_{\mathclap{\forall v \in V(G)}}d(v) = 2\abs{E(G)}
	\end{equation*}
	\begin{proof}
		Với mỗi $e \in E(G)$, $e$ có hai đầu mút, làm tăng bậc của $2$ đỉnh đầu mút của nó lên 1 và do đó tăng tổng số bậc của đồ thị lên 2.
	\end{proof}
\end{theorem}
\begin{corolarry}
	\label{coro:number_of_odd_degree_vertices_is_even}
Số đỉnh bậc lẻ trong một đồ thị luôn là số chẵn. Không có đồ thị chính quy nào có bậc lẻ.
\end{corolarry}
\begin{corolarry}
	Đồ thị $k$ -- chính quy với $n$ đỉnh có $nk/2$ cạnh.
\end{corolarry}

\begin{proposition}
	\label{prop:number_of_n_vertice_graph} Số đồ thị đơn giản với tập đỉnh có $n$ phần tử là $2^{C_n^2}$	
	\begin{proof}
		Gọi $V$ là tập đỉnh có $n$ phần tử. Ta xây dựng các đồ thị giản đơn $G$ từ tập đỉnh $V$. Có $C^2_n$ cách chọn một cặp cạnh ở trong $V$. Với mỗi cặp cạnh, ta có 2 lựa chọn: kề nhau hoặc không kề nhau. Do đó có tổng cộng $2^{C_n^2}$ đồ thị đơn giản với $n$ đỉnh.
	\end{proof}
\end{proposition}

\begin{proposition}
	Với $n > 2$, có $2^{C_{n-1}^2}$ đồ thị đơn giản có các đỉnh $v_1$, $v_2$, $v_3, \ldots, v_n$ mà bậc của mỗi đỉnh đều chẵn.
	\begin{proof}
		Gọi tập $A$ là tập các đồ thị đơn giản mà có các đỉnh là $v_1$, $v_2$, $v_3, \ldots, v_{n-1}$, $B$ là tập các đồ thị đơn giản với các đỉnh $v_1$, $_2$, $_3,\ldots,v_n$. Theo mệnh đề \ref{prop:number_of_n_vertice_graph} ta có $\abs{A} = 2^{C_{n-1}^2}$. Ta định nghĩa ánh xạ $f:A\to B$: $f$ biến đồ thị $G \in A$ thành $H \in B$ bằng cách thêm một đỉnh $v_n$, sau đó nối $v_n$ với các đỉnh bậc lẻ của $G$. Khi đó các đỉnh bậc lẻ của $G$ cũng sẽ trở thành các đỉnh bậc chẵn trong $H$, và theo hệ quả \ref{coro:number_of_odd_degree_vertices_is_even}, bản thân $v_n$ cũng là bậc chẵn. Ngược lại, nếu lấy đồ thị $H$ bất kì thuộc $B$ và ngắt bỏ đỉnh $v_n$, ta thu được một đồ thị $G$ tương ứng thuộc $A$. Dễ thấy ngay quá trình này là nghịch đảo của quá trình trên, do đó $f$ là một song ánh. Vì $f$ là song ánh nên $\abs{B} = \abs{A} = 2^{C_{n-1}^2}$
	\end{proof}
\end{proposition}

\begin{proposition}
	Đồ thị đơn giản mà có nhiều hơn một đỉnh thì có hai đỉnh với bậc bằng nhau.
	\begin{proof}
		Trong đồ thị với $n$ đỉnh, bậc của các đỉnh nhận giá trị trong tập $\left\{0,1,2,3,\ldots,n-1\right\}$. Tuy nhiên, giá trị $0$ và $n-1$ sẽ không xuất hiện cùng một lúc vì đỉnh có bậc $n-1$ sẽ kề với mọi đỉnh, đỉnh có bậc $0$ là cô lập, 2 đỉnh như trên không thể cùng xuất hiện trong một đồ thị. Do đó, số giá trị bậc các đỉnh của $G$ luôn bé hơn $n$, theo nguyên líDirichlet, mệnh đề được chứng minh.
	\end{proof}
\end{proposition}
\begin{proposition}
	Nếu $G$ là đồ thị đơn giản với $n$ đỉnh và $\delta(G) \ge (n-1)/2$ thì $G$ liên thông.
	\begin{proof}
		Lấy 2 đỉnh $u,v \in V(G)$. Giả sử $u,v$ không kề nhau. Vì $\delta(G) \ge (n-1)/2$, sẽ có ít nhất $n-1$ cạnh để nối $u$ hoặc $v$ tới các đỉnh khác. Tuy nhiên chỉ có $n-2$ cạnh, do đó theo nguyên lí Dirichlet, sẽ có một đỉnh nào đó kề với cả $u$ và $v$. Vậy mỗi cặp đỉnh trong $G$ đều tồn tại một đường đi giữa chúng.
	\end{proof}
\end{proposition}
\begin{theorem}
	\label{theo:sub_bigraph}
	Cho đồ thị $G$ không có khuyên. Khi đó $G$ có một đồ thị con là đồ thị hai phía với ít nhất $\abs{E(G)}/2$ cạnh
	\begin{proof}
		Phân hoạch tập đỉnh của $G$ thành hai tập $V_1$ và $V_2$. Lấy các cạnh $e=uv$ sao cho $u \in V_1$ và $v\in V_2$ đưa vào tập $E(H)$. Khi đó ta thu được đồ thị $H = (V_1,V_2,E(H))$ là đồ thị con 2 phía của $G$. Với mọi đỉnh $v$ có bậc $a$ trong $H$ và bậc $b$ trong $G$ mà $a < b/2$, nếu $v \in V_1$ thì ta chuyển $v$ sang $V_2$ và ngược lại. Khi đó ta thu được đồ thị mới $H'$ thỏa mãn mệnh đề trên. Lưu ý rằng không nhất thiết phải chuyển mọi đỉnh $v$ như trên mà chỉ cần chuyển tới khi thu được đồ thị cần tìm.
	\end{proof}
\end{theorem}

\begin{figure}[htpb]
	\begin{center}
		\begin{minipage}[b]{0.3\textwidth}
		\begin{center}
		\begin{tikzpicture}[thick]
			\node[vertice, label={225:$v_1$}] (v1) at ( 0,0) {};	
			\node[vertice, label={-45:$v_2$}] (v2) at ( 1,0) {};	
			\node[vertice, label={360:$v_3$}] (v3) at ( 2,1) {};	
			\node[vertice, label={045:$v_4$}] (v4) at ( 1,2) {};	
			\node[vertice, label={135:$v_5$}] (v5) at ( 0,2) {};	
			\node[vertice, label={180:$v_6$}] (v6) at (-1,1) {};	
			\path[draw=black] (v1)--(v2)--(v3)--(v4)--(v5)--(v6)--(v1)--(v4)(v1)--(v3);
		\end{tikzpicture}\\$G$
		\end{center}
		\end{minipage}
		\begin{minipage}[b]{0.3\textwidth}
		\begin{center}
		\begin{tikzpicture}[thick]
			\node[vertice, label={-90:$v_1$}, fill=white] (v1) at (0,0) {};	
			\node[vertice, label={-90:$v_2$}, fill=white] (v2) at (1,0) {};	
			\node[vertice, label={-90:$v_3$}, fill=white] (v3) at (2,0) {};	
			\node[vertice, label={090:$v_4$}] (v4) at (0,2) {};	
			\node[vertice, label={090:$v_5$}] (v5) at (1,2) {};	
			\node[vertice, label={090:$v_6$}] (v6) at (2,2) {};	
			\path[draw=black] (v1)(v2)(v3)--(v4)(v5)(v6)--(v1)--(v4)(v1)(v3);
		\end{tikzpicture}\\$H$
		\end{center}
		\end{minipage}
		\begin{minipage}[b]{0.3\textwidth}
		\begin{center}
		Chuyển cạnh $v_5$ sang phía còn lại:
		\begin{tikzpicture}[thick]
			\node[vertice, label={-90:$v_1$}, fill=white] (v1) at (0,0) {};	
			\node[vertice, label={-90:$v_2$}, fill=white] (v2) at (1,0) {};	
			\node[vertice, label={-90:$v_3$}, fill=white] (v3) at (2,0) {};	
			\node[vertice, label={-90:$v_5$}, fill=white] (v5) at (3,0) {};	
			\node[vertice, label={090:$v_4$}] (v4) at (1,2) {};	
			\node[vertice, label={090:$v_6$}] (v6) at (2,2) {};	
			\path[draw=black] (v1)(v2)(v3)--(v4)--(v5)--(v6)--(v1)--(v4)(v1)(v3);
		\end{tikzpicture}\\$H'$
		\end{center}
		\end{minipage}
	\end{center}
	\caption{Minh họa cho định lý \ref{theo:sub_bigraph}}
	\label{fig:minh_hoa_cho_dinh_ly_sub_bigraph}
\end{figure}


