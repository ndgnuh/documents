\documentclass[14pt]{extarticle}
\usepackage[vietnamese]{babel}
\usepackage[utf8]{inputenc}
\usepackage{amsmath}
\usepackage{amsfonts}
\usepackage{amssymb}
\usepackage{amsthm}
\usepackage[top=2cm,left=3cm,right=2cm,bottom=2cm,a4paper]{geometry}
\usepackage{tikz}
\usepackage{parskip}
\usepackage{mathtools}

\usetikzlibrary{arrows}
\tikzstyle{vertice}=[circle, draw=black, fill=cyan, inner sep=0.15em]
\setlength{\parindent}{0pt}
\linespread{1.15}

\newtheorem{definition}{Định nghĩa}[section]
\newtheorem{example}{Ví dụ}[section]
\newtheorem{theorem}{Định lý}[section]
\newtheorem{lemma}{Bổ đề}[section]

\tikzstyle{vertice}=[circle, draw=black, fill=cyan, inner sep=0.15em]


\begin{document}
% \tableofcontents\pagebreak
\section{Đồ thị}
\begin{definition}
	Đồ thị $G$ là một cấu trúc rời rạc gồm các thành phần là tập ${V(G) = \{v_1,v_2,\cdots,v_n\}}$, $E(G) = \{e_1,e_2,\cdots,e_m\}$, được gọi một cách tương ứng là tập đỉnh và tập cạnh của đồ thị. Ký hiệu: $G = (V, E)$.
\end{definition}
Một số khái niệm liên quan:
\begin{itemize}
	\item Số cạnh của đồ thị gọi là \textit{bậc của đồ thị}, kí hiệu: $n(G)$ (hoặc $n$).
	\item Mỗi phần tử thuộc $V(G)$ được gọi là một \textit{đỉnh} của $G$. Một phần tử thuộc $E(G)$ được gọi là một \textit{cạnh} của $G$.
	\item Một cạnh của $G$ sẽ nối hai đỉnh của $G$. Nếu cạnh $v,u \in V(G)$ có cạnh nối giữa chúng thì nói $u$ \textit{kề} $v$. Cạnh được kí hiệu bằng một cặp đỉnh: $e = (u,v)$, khi đó $u,v$ gọi là \textit{đầu mút} của $e$.
	\item \textit{Khuyên} là cạnh của đồ thị mà có hai đầu mút cùng là một đỉnh.
	\item \textit{Hai cạnh song song} (hay còn gọi là \textit{cạnh đôi}) là hai cạnh mà có chung cặp đầu mút.
	\item \textit{Bậc của đỉnh} $v$ là tổng số cạnh mà có $u$ là đầu mút. Nếu $v$ là đầu mút của một khuyên thì khuyên đó tính là 2 cạnh. Kí hiệu: $d(v)$
	\item \textit{Bậc lớn nhất} trong $G$: $\Delta = \max\limits_{v\in V} \{d(v)\}$
	\item \textit{Bậc nhỏ nhất} trong $G$: $\delta = \min\limits_{v\in V} \{d(v)\}$
\end{itemize}
\begin{figure}[ht]
	\centering
	\hfill
	\begin{tikzpicture}
		\node[vertice](v1) at (1,0) {};
		\node[vertice](v2) at (1,2) {};
		\node[vertice](v3) at (3,1.2) {};
		\node[vertice](v4) at (0.3,1) {};
		\node[vertice](v5) at (1.22,0.9) {};
		\path[draw=black] 
			(v1) to[in=-120, out=120] (v2)
			(v3) to[in=30, out=100] (v4)
			(v2) to[out=90, in=180] (v4)
			(v5) arc[start angle=180, end angle=-180, radius=1em];
	\end{tikzpicture}
	\hfill
	\begin{tikzpicture}[scale=1, transform shape]
		\node[vertice](v1) at (0,0) {};	
		\node[vertice](v2) at (2,0) {};	
		\node[vertice](v3) at (0,2) {};	
		\node[vertice](v4) at (2,2) {};	
		\path[draw=black]
			(v2)--(v4)--(v3)--(v1)--(v2)--(v3);
	\end{tikzpicture}
	\hfill
	\begin{tikzpicture}
		\node[vertice](v1) at (0,0) {};
		\node[vertice](v2) at (2,0) {};
		\node[vertice](v3) at (1,2) {};
		\node[vertice](v4) at (1,0.7) {};
		\path[draw=black](v1)--(v2)--(v3)--(v1)--(v4)--(v2)--(v3)(v4)--(v3);
	\end{tikzpicture}
	\hfill~
	\caption{Minh họa một số đồ thị}
\end{figure}

\begin{definition}[Đồ thị có hướng]
	Đồ thị $G=(V,E)$ là đồ thị có hướng $\iff \forall (u,v)\in E,\ (u,v)$ sắp thứ tự. Khi đó $u$ gọi là đỉnh ra, $v$ là đỉnh vào, cạnh $(u,v)$ gọi là một cung.
\end{definition}
\begin{figure}
\begin{center}
\begin{tikzpicture}[->, >=stealth, scale=1, transform shape]
	\node[vertice] (v1) at (0,2) {};
	\node[vertice] (v2) at (1,0) {};
	\node[vertice] (v3) at (3,0) {};
	\node[vertice] (v4) at (2,2) {};
	\draw (v1) -> (v2);
	\draw (v2) -> (v3);
	\draw (v3) -> (v4);
	\draw (v2) -> (v4);
	\draw (v4) -> (v1);
\end{tikzpicture}
\end{center}
\caption{Đồ thị có hướng}
\label{fig:do-thi-co-huong}
\end{figure}



\section{Ma trận biểu diễn và đồng cấu}
\section{Đường đi và chu trình}
\section{Bậc của đỉnh}
\section{Các phép toán trên đồ thị}
\section{Một số kĩ thuật cơ bản}
\section{Chuỗi bậc}
\end{document}
