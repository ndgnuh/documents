\documentclass[14pt]{extarticle}
\usepackage[vietnamese]{babel}
\usepackage[utf8]{inputenc}
\usepackage{amsmath}
\usepackage{amsfonts}
\usepackage{amssymb}
\usepackage{amsthm}
\usepackage[top=2cm,left=3cm,right=2cm,bottom=2cm,a4paper]{geometry}
\usepackage{tikz}
\usepackage{parskip}
\usepackage{mathtools}

% \let\oldforall\forall
% \let\forall\undefined
% \DeclareMathOperator{\forall}{\oldforall}
% \newcommand{\forall}{\oldforall\ }

\usetikzlibrary{arrows}
\tikzstyle{vertice}=[circle, draw=black, fill=black, inner sep=1mm]
\setlength{\parindent}{0pt}
\linespread{1.15}

\newtheorem{definition}{Định nghĩa}[section]
\newtheorem{example}{Ví dụ}[section]
\newtheorem{theorem}{Định lý}[section]
\newtheorem{lemma}{Bổ đề}[section]
\newtheorem{proposition}{Mệnh đề}[section]
\newtheorem{corolarry}{Hệ quả}[section]

\newcommand{\abs}[1]{\left\lvert#1\right\rvert}


\begin{document}
\tableofcontents\pagebreak

% \hspace{1cm}\vfill
% \begin{center}
% 	\textit{Trong bài viết, những định nghĩa, khái niệm, hoặc có chung biểu diễn tiếng Việt với nhau, hoặc chưa có từ tiếng Việt tương ứng, sẽ được giữ nguyên gốc tiếng Anh.}
% \end{center}
% \vfill
% \pagebreak

\section{Đồ thị}

\begin{definition}
	Đồ thị $G$ là một cấu trúc rời rạc gồm các thành phần là tập ${V(G) = \{v_1,v_2,\cdots,v_n\}}$, $E(G) = \{e_1,e_2,\cdots,e_m\}$, được gọi một cách tương ứng là tập đỉnh và tập cạnh của đồ thị. Ký hiệu: $G = (V, E)$.
\end{definition}
Một số khái niệm liên quan:
\begin{itemize}
	\item Số cạnh của đồ thị gọi là \textit{bậc của đồ thị}, kí hiệu: $n(G)$ (hoặc $n$).
	\item Mỗi phần tử thuộc $V(G)$ được gọi là một \textit{đỉnh} của $G$. Một phần tử thuộc $E(G)$ được gọi là một \textit{cạnh} của $G$.
	\item Một cạnh của $G$ sẽ nối hai đỉnh của $G$. Nếu cạnh $v,u \in V(G)$ có cạnh nối giữa chúng thì nói $u$ \textit{kề} $v$. Cạnh được kí hiệu bằng một cặp đỉnh: $e = (u,v)$, khi đó $u,v$ gọi là \textit{đầu mút} của $e$.
	\item \textit{Khuyên} là cạnh của đồ thị mà có hai đầu mút cùng là một đỉnh.
	\item \textit{Hai cạnh song song} (hay còn gọi là \textit{cạnh đôi}) là hai cạnh mà có chung cặp đầu mút.
	\item \textit{Bậc của đỉnh} $v$ là tổng số cạnh mà có $u$ là đầu mút. Nếu $v$ là đầu mút của một khuyên thì khuyên đó tính là 2 cạnh. Kí hiệu: $d(v)$
	\item \textit{Bậc lớn nhất} trong $G$: $\Delta = \max\limits_{v\in V} \{d(v)\}$
	\item \textit{Bậc nhỏ nhất} trong $G$: $\delta = \min\limits_{v\in V} \{d(v)\}$
	\item \textit{Đường đi và chu trình:} (định nghĩa \ref{def:duong-di} và \ref{def:chu-trinh})
\end{itemize}

\begin{figure}[ht]
	\centering
	\hfill
	\begin{tikzpicture}
		\node[vertice](v1) at (1,0) {};
		\node[vertice](v2) at (1,2) {};
		\node[vertice](v3) at (3,1.2) {};
		\node[vertice](v4) at (0.3,1) {};
		\node[vertice](v5) at (1.22,0.9) {};
		\path[draw=black] 
			(v1) to[in=-120, out=120] (v2)
			(v3) to[in=30, out=100] (v4)
			(v2) to[out=90, in=180] (v4)
			(v5) arc[start angle=180, end angle=-180, radius=1em];
	\end{tikzpicture}
	\hfill
	\begin{tikzpicture}[scale=1, transform shape]
		\node[vertice](v1) at (0,0) {};	
		\node[vertice](v2) at (2,0) {};	
		\node[vertice](v3) at (0,2) {};	
		\node[vertice](v4) at (2,2) {};	
		\path[draw=black]
			(v2)--(v4)--(v3)--(v1)--(v2)--(v3);
	\end{tikzpicture}
	\hfill
	\begin{tikzpicture}
		\node[vertice](v1) at (0,0) {};
		\node[vertice](v2) at (2,0) {};
		\node[vertice](v3) at (1,2) {};
		\node[vertice](v4) at (1,0.7) {};
		\path[draw=black](v1)--(v2)--(v3)--(v1)--(v4)--(v2)--(v3)(v4)--(v3);
	\end{tikzpicture}
	\hfill~
	\caption{Minh họa một số đồ thị}
	\label{fig:minh-hoa-mot-so-do-thi}
\end{figure}


\begin{definition}[Đồ thị có hướng]
	Đồ thị $G=(V,E)$ là đồ thị có hướng $\iff \forall (u,v)\in E,\ (u,v)$ sắp thứ tự. Khi đó $u$ gọi là đỉnh ra, $v$ là đỉnh vào, cạnh $(u,v)$ gọi là một cung.
\end{definition}
\begin{figure}[htpb]
\begin{center}
\begin{tikzpicture}[thick, ->, >=stealth, scale=1, transform shape]
	\node[vertice] (v1) at (0,2) {};
	\node[vertice] (v2) at (1,0) {};
	\node[vertice] (v3) at (3,0) {};
	\node[vertice] (v4) at (2,2) {};
	\draw (v1) -> (v2);
	\draw (v2) -> (v3);
	\draw (v3) -> (v4);
	\draw (v2) -> (v4);
	\draw (v4) -> (v1);
\end{tikzpicture}
\end{center}
\caption{Đồ thị có hướng}
\label{fig:do-thi-co-huong}
\end{figure}


\begin{definition}
	[Đồ thị có trọng số] Là đồ thị mà mỗi cạnh của nó được gắn với một trọng số.
\end{definition}
\begin{definition}
	[Đồ thị con] Đồ thị $H$ gọi là đồ thị con của $G \iff V(H) \subseteq V(G) \land E(H) \subseteq E(G)$. Kí hiệu:$H \subseteq G$. Nếu $H \subseteq G$ và $V(H) = V(G) = V$ thì ta gọi $H$ là đồ thị con mở rộng của $G$.
\end{definition}

\begin{definition}
	[Đồ thị đầy đủ/clique] Đồ thị $G$ là đồ thị đầy đủ (clique) $\iff$ mọi đỉnh của $G$ đều có cạnh nối giữa chúng. Đồ thị đầy đủ có $n$ đỉnh được kí hiệu là $K_n$. Đồ thị $\overline G$ được gọi là phủ của $G \iff$ $V(G)= V(\overline G)$ và $V(\overline G) = V(K_n) \setminus V(G)$
\end{definition}
\begin{figure}[htpb]
\begin{center}
\hfill
\begin{tikzpicture}[thick]
	\node[vertice](v1) at (0,0) {};
	\node[vertice](v2) at (2,0) {};
	\node[vertice](v3) at (2,2) {};
	\node[vertice](v4) at (0,2) {};
	\path[draw=black] (v1) -- (v2) (v1)--(v3) (v1)--(v4);
\end{tikzpicture}\hfill
\begin{tikzpicture}[thick]
	\node[vertice](v1) at (0,0) {};
	\node[vertice](v2) at (2,0) {};
	\node[vertice](v3) at (2,2) {};
	\node[vertice](v4) at (0,2) {};
	\path[draw=black] (v2)--(v3)--(v4)--(v2);
\end{tikzpicture}\hfill~
\end{center}
\caption{Đồ thị $G$ và $\overline G$}
\label{fig:do-thi-g-va-g-ngang}
\end{figure}



\begin{definition}
	[Đồ thị hai phía] Đồ thị $G$ là đồ thị 2 phía $\iff V(G) = V_1 \cup V_2$ với $V_1, V_2$ là tập độc lập, $V_1 \cap V_2 = \varnothing$. Nếu $\forall u \in V_1, v \in V_2, v \textnormal{ kề } u$ thì $G$ gọi là đồ thị hai phía đầy đủ và được kí hiệu là $K_{m,n}$ (với $m = \abs{V_1}, n = \abs{V_2}$).	
\end{definition}
\begin{figure}[htpb]
\begin{center}
\begin{tikzpicture}[thick]
	\node[vertice, fill=white] (v1) at (0,0){};	
	\node[vertice, fill=white] (v2) at (3,0){};	
	\node[vertice, fill=white] (v3) at (6,0){};	
	\node[vertice] (v4) at (1.5,3){};	
	\node[vertice] (v5) at (4.5,3){};	
	\path[draw=black] (v1) -- (v4) -- (v2) -- (v5) -- (v3);
	\path[draw=black] (v1) -- (v5) -- (v2) -- (v4) -- (v3);
\end{tikzpicture}
\end{center}
\caption{Đồ thị hai phía đầy đủ $K_{2,3}$}
\label{fig:do-thi-hai-phia-day-du-k23}
\end{figure}



\begin{definition}
	[Tập độc lập] Tập độc lập trong đồ thị $G$ được định nghĩa bởi ${S = \{v \in V(G) \mid \textnormal{không có cặp đỉnh nào kề nhau} \}}$
\end{definition}

\begin{definition}
	[Đồ thị liên thông] Đồ thị $G$ gọi là liên thông $\iff \forall\ u,v \in V,\ \exists$ đường đi từ $u$ tới $v$. Ngược lại, nếu $\exists\ u,v \in V,\ \nexists$ đường đi từ $u$ tới $v$ thì $G$ là đồ thị \textbf{không} liên thông.
\end{definition}
\begin{figure}[htpb]
\begin{center}
\begin{tikzpicture}[thick]
	\node[vertice, fill=white] (v1) at (0,0){};	
	\node[vertice, fill=white] (v3) at (6,0){};	
	\node[vertice, fill=white] (v5) at (3,3){};	
	\node[vertice] (v2) at (3,0){};	
	\node[vertice] (v4) at (0,3){};	
	\node[vertice] (v6) at (6,3){};	
	\path[draw=black] (v1) -- (v5) -- (v3);
	\path[draw=black] (v4) -- (v2) -- (v6);
\end{tikzpicture}
\end{center}
\caption{Đồ thị không liên thông}
\label{fig:do-thi-khong-lien-thong}
\end{figure}




\section{Ma trận biểu diễn và đẳng cấu}
Đồ thị $G$ có thể được biểu diễn bằng một ma trận $A(G) = [a_{ij}]$ với $a_{ij}$ là số cạnh nối nữa $v_i$ và $v_j$. Nếu $G$ là đồ thị có hướng, $a_{ij}$ sẽ là số cung đi từ $v_i$ tới $v_j$.

Nếu cạnh $e$ có đầu mút là đỉnh $v$ thì $e$ gọi là cạnh kề của $v$. Đồ thị G cũng có thể biểu diễn bằng ma trận $M = [m_{ij}]$, trong đó $m_{ij} = 1$ nếu đỉnh $v_i$ có cạnh kề $e_j$. Trong trường hợp $G$ là đồ thị có hướng, $m_{ij} = 1$ nếu $v_i$ là đỉnh ra  và $-1$ nếu $v_i$ là đỉnh vào của $e_j$. Nếu không rơi vào những trường hợp trên thì $m_{ij} = 0$.


\begin{figure}[h]
	\begin{center}
		\begin{minipage}[c]{0.3\textwidth}
			\begin{tikzpicture}[thick]
				\node[vertice, label={[label distance=1mm]120:1}] (v1) at (0,0) {};
				\node[vertice, label={[label distance=1mm]30:2}] (v2) at (3,0) {};
				\node[vertice, label={[label distance=1mm]30:3}] (v3) at (3,3) {};
				\node[vertice, label={[label distance=1mm]120:4}] (v4) at (0,3) {};
				\path[draw=black] (v1) 
					--node[anchor=east]{a} (v4) 
					--node[anchor=west]{b} (v2) 
					--node[anchor=west]{c} (v3);
			\end{tikzpicture}
		\end{minipage}
		\begin{minipage}[c]{0.3\textwidth}
			\begin{align*}
				\left(
					\begin{array}{c|cccc}
						&1&2&3&4\\\hline
						1&0&0&0&1\\
						2&0&0&1&1\\
						3&0&1&0&0\\
						4&1&1&0&0
					\end{array}
				\right)
			\end{align*}
		\end{minipage}
		\begin{minipage}[c]{0.3\textwidth}
			\begin{align*}
				\left(
					\begin{array}{c|ccc}
						 &a&b&c\\\hline
						1&1&0&0\\
						2&0&1&1\\
						3&0&0&1\\
						4&1&1&0
					\end{array}
				\right)
			\end{align*}
		\end{minipage}\par
		\begin{minipage}{0.3\textwidth}\begin{center}$G$\end{center}\end{minipage}
		\begin{minipage}{0.3\textwidth}\begin{center}$A(G)$\end{center}\end{minipage}
		\begin{minipage}{0.3\textwidth}\begin{center}$M(G)$\end{center}\end{minipage}
	\end{center}
	\caption{Biểu diễn đồ thị $G$}
	\label{fig:bieu-dien-do-thi-g}
\end{figure}

\begin{figure}[ht]
	\begin{center}
		\begin{minipage}[c]{0.3\textwidth}
			\begin{tikzpicture}[->, >=stealth, thick]
				\node[vertice, label={[label distance=1mm]120:1}] (v1) at (0,0) {};
				\node[vertice, label={[label distance=1mm]30:2}] (v2) at (3,0) {};
				\node[vertice, label={[label distance=1mm]30:3}] (v3) at (3,3) {};
				\node[vertice, label={[label distance=1mm]120:4}] (v4) at (0,3) {};
				\draw (v1)--node[anchor=east]{a} (v4);
				\draw (v4)--node[anchor=west]{b} (v2);
				\draw (v2)--node[anchor=west]{c} (v3);
			\end{tikzpicture}
		\end{minipage}
		\begin{minipage}[c]{0.3\textwidth}
			\begin{align*}
				\left(
					\begin{array}{c|cccc}
						&1&2&3&4\\\hline
						1&0&0&0&1\\
						2&0&0&1&0\\
						3&0&0&0&0\\
						4&0&1&0&0
					\end{array}
				\right)
			\end{align*}
		\end{minipage}
		\begin{minipage}[c]{0.3\textwidth}
			\begin{align*}
				\left(
					\begin{array}{c|rrr}
						 &a&b&c\\\hline
						1&+1& 0& 0\\
						2& 0&-1&+1\\
						3& 0& 0&-1\\
						4&-1&+1& 0
					\end{array}
				\right)
			\end{align*}
		\end{minipage}\par
		\begin{minipage}{0.3\textwidth}\begin{center}$G$\end{center}\end{minipage}
		\begin{minipage}{0.3\textwidth}\begin{center}$A(G)$\end{center}\end{minipage}
		\begin{minipage}{0.3\textwidth}\begin{center}$M(G)$\end{center}\end{minipage}
	\end{center}
	\caption{Biểu diễn đồ thị $G$ (có hướng)}
	\label{fig:bieu-dien-do-thi-g-co-huong}
\end{figure}



Một đồ thị có thể có nhiều ma trận $A(G)$ hoặc $M(G)$ khác nhau, tùy vào cách đánh số cạnh/đỉnh. Sau đây là một số tính chất:
\begin{itemize}
	\item Nếu $A(G)$ là ma trận đối xứng thì đồ thị $G$ vô hướng. 
	\item Nếu $A(G)$ đối xứng, $a_{ij} \in \{0,1\}$ và $a_{ii} = 0$ thì $G$ là đồ thị đơn giản.
\end{itemize}



\section{Đường đi và chu trình}
\begin{definition}[đường đi]
	\label{def:duong-di}
	Một \textbf{đường đi} (kí hiệu W) có độ dài $k$ là một dãy các \textbf{đỉnh và cạnh} $v_0, e_1, v_1, e_2,\cdots,e_k,v_k$ với $e_i = (v_{i-1},v_i)$. Độ dài của một đường đi kí hiệu là $l(W)$. Đỉnh $v_0$ gọi là đỉnh đầu và $v_k$ gọi là đỉnh cuối. Nếu không có cạnh nào lặp lại, đường đi gọi là đơn. Nếu không có đỉnh nào lặp lại, đường đi gọi là sơ cấp.
\end{definition}
\begin{definition}[chu trình]
	\label{def:chu-trinh}
	 Một chu trình là một đường đi đơn mà có đỉnh đầu và đỉnh cuối trùng nhau. Nếu chỉ có đỉnh đầu và đỉnh cuối trùng nhau thì chu trình gọi là sơ cấp.
\end{definition}
Có thể thấy chu trình với độ dài 1 là một khuyên.
\begin{proposition}
	Đồ thị $G$ là liên thông $\iff ({\exists uv \in E})({ u\in V_1})({v \in V_2})$ với mọi $V_1,\,V_2 \ne \varnothing$ mà $V = V_1\cup V_2,\, V_1 \cap V_2 = \varnothing$ ($V_1,\,V_2$ là phân hoạch không rỗng của $V$)
	\begin{proof}
		Giả sử $G$ liên thông. Khi đó tồn tại một đường đi sơ cấp từ $u$ tới $v$, trên đường đi đó, sau đỉnh cuối cùng của $V_1$ mà nó đi qua là cạnh nối giữa $V_1$ và $V_2$. Giả sử $G$ không liên không. Chứng minh được chiều xuôi.\\
		Gọi $H$ là một thành phần liên thông của $G$, chọn $V_1 = V(H)$. Khi đó không có cạnh nào của $G$ có một đầu mút thuộc $V_1$ và đầu mút còn lại thuộc $V_2$. Bằng cách đảo vế chứng minh được chiều ngược.
	\end{proof}
\end{proposition}

\begin{theorem}
Một cạnh của $G$ là cạnh cắt $\iff$ cạnh đó không thuộc chu trình sơ cấp nào.	
\begin{proof}
	Gọi $e=(uv)$ là một cạnh trong thành phần liên thông $H$ của $G$. Nếu đồ thị thu được từ $H$ bỏ đi $e$ (kí hiệu $H-e$) vẫn liên thông, thì tồn tại một đường đi sơ cấp từ $u$ tới $v$. Có thể thấy đường đi này nếu gắn thêm cạnh $e$ sẽ tạo thành một chu trình sơ cấp. Vậy nếu $e$ không phải cạnh cắt thì sẽ thuộc một chu trình sơ cấp. Nếu $e$ là cạnh cắt, $H$ bỏ đi $e$ không liên thông, không tồn tại đường đi từ $u$ tới $v$. (TBA?)
\end{proof}
\end{theorem}

\begin{proposition}
	Cho đồ thị $G = (V,E)$, $V=\{v_1, v_2,\ldots,v_n\}$ với $n \ge 3$. Nếu có ít nhất 2 trong các đồ thị con $G-v_1,G-v_2,\ldots,G-v_n$ ($G$ xóa đi đỉnh $v_i$) liên thông thì $G$ liên thông.
	\begin{proof}
		Giả sử $G$ không liên thông, $H_i,\, i = \overline{1,k}$ là các thành phần liên thông của $G$. Nếu xóa một đỉnh khỏi $H_i$ thì không làm thay đổi số thành phần liên thông, trừ khi $H_i = K_1$ (giảm đi còn $k-1$ thành phần liên thông). Nếu 2 trong số $G-v_1,G-v_2,\ldots,G-v_n$ liên thông thì $k = 2$ và $H_1 = H_2 = K_1$, mâu thuẫn với $n \ge 3$.
	\end{proof}
\end{proposition}

\begin{corolarry} Mọi đồ thị $G$ mà chứa ít nhất một cạnh thì có ít nhất hai đỉnh không phải là đỉnh cắt\end{corolarry}
\begin{theorem}
Một đồ thị là một đồ thị hai phía $\iff$ nó không chứa chu trình độ dài lẻ.	
\begin{proof}
	Giả sử $G$ là một đồ thị hai phía, khi đó mọi đường đi trên $G$ đều chuyển qua chuyển lại giữa các đỉnh thuộc $V_1$ và $V_2$. Vì vậy, nếu muốn quay lại điểm bắt đầu thì đường đi buộc phải có độ dài chẵn.\\
	Giả sử $G$ không có chu trình độ dài chẵn. $G$ liên thông. Chọn cố định điểm $u$ thuộc $V(G)$. Với mọi điểm $v \in V(G)$, mọi đường đi từ $u$ tới $v$ đều phải có độ dài cùng chẵn hoặc cùng lẻ, nếu không sẽ tạo ra một chu trình độ dài lẻ (mâu thuẫn). Ta phân hoạch $V(G)$ thành 2 tập $V_1 = \{v\mid$ đường đi từ $u$ tới $v$ độ dài lẻ $\}$ và $V_2 = \{v\mid$ đường đi từ $u$ tới $v$ độ dài chẵn $\}$. Thấy được $V_1$ là một tập độc lập vì một cạnh nối giữa hai đỉnh của $V_1$ sẽ tạo ra một chu trình độ dài lẻ. Ta có điều tương tự với $V_2$. Do đó $G$ là một đồ thị hai phía.
\end{proof}
\end{theorem}
\section{Bậc của các đỉnh trong đồ thị}
\label{sec:bac_cua_dinh}
\begin{theorem}
	\label{theo:tong_bac_cua_dinh} Cho một đồ thị $G$, tổng bậc của các đỉnh trong $G$ bằng 2 lần số cạnh của $G$.
	\begin{equation*}
		\sum_{\mathclap{\forall v \in V(G)}}d(v) = 2\abs{E(G)}
	\end{equation*}
	\begin{proof}
		Với mỗi $e \in E(G)$, $e$ có hai đầu mút, làm tăng bậc của $2$ đỉnh đầu mút của nó lên 1 và do đó tăng tổng số bậc của đồ thị lên 2.
	\end{proof}
\end{theorem}
\begin{corolarry}
Số đỉnh bậc lẻ trong một đồ thị luôn là số chẵn. Không có đồ thị chính quy nào có bậc lẻ.
\end{corolarry}
\begin{corolarry}
	Đồ thị $k$ -- chính quy với $n$ đỉnh có $nk/2$ cạnh.
\end{corolarry}

\section{Một số kĩ thuật trên đồ thị}
\begin{proposition}
	Với $n > 2$, có $2^{C_{n-1}^2}$ đồ thị đơn giản với tập đỉnh $v_1, v_2, v_3, \ldots, v_n$ mà bậc của mỗi đỉnh đều chẵn.
\end{proposition}
\begin{proposition}
	Đồ thị đơn giản mà có nhiều hơn một đỉnh thì có hai đỉnh với bậc bằng nhau.
	\begin{proof}
		Trong đồ thị với $n$ đỉnh, bậc của các đỉnh nhận giá trị trong tập $\left\{0,1,2,3,\ldots,n-1\right\}$. Tuy nhiên, giá trị $0$ và $n-1$ sẽ không xuất hiện cùng một lúc vì đỉnh có bậc $n-1$ sẽ kề với mọi đỉnh, đỉnh có bậc $0$ là cô lập, 2 đỉnh như trên không thể cùng xuất hiện trong một đồ thị. Do đó, số giá trị bậc các đỉnh của $G$ luôn bé hơn $n$, theo nguyên líDirichlet, mệnh đề được chứng minh.
	\end{proof}
\end{proposition}
\begin{proposition}
	Nếu $G$ là đồ thị đơn giản với $n$ đỉnh và $\delta(G) \ge (n-1)/2$ thì $G$ liên thông.
	\begin{proof}
		Lấy 2 đỉnh $u,v \in V(G)$. Giả sử $u,v$ không kề nhau. Vì $\delta(G) \ge (n-1)/2$, sẽ có ít nhất $n-1$ cạnh để nối $u$ hoặc $v$ tới các đỉnh khác. Tuy nhiên chỉ có $n-2$ cạnh, do đó theo nguyên lí Dirichlet, sẽ có một đỉnh nào đó kề với cả $u$ và $v$. Vậy mỗi cặp đỉnh trong $G$ đều tồn tại một đường đi giữa chúng.
	\end{proof}
\end{proposition}
\begin{theorem}
	Cho đồ thị $G$ không có khuyên. Khi đó $G$ có một đồ thị con là đồ thị hai phía với ít nhất $\abs{E(G)}/2$ cạnh
	\begin{proof}
		Phân hoạch tập đỉnh của $G$ thành hai tập $V_1$ và $V_2$. Lấy các cạnh $e=uv$ sao cho $u \in V_1$ và $v\in V_2$ đưa vào tập $E(H)$. Khi đó ta thu được đồ thị $H = (V_1,V_2,E(H))$ là đồ thị con 2 phía của $G$. Giả sử cạnh $v \in V(H)$ có ít hơn $\abs{E(G)}/2$ cạnh nối với nó. (tba)<++>
	\end{proof}
\end{theorem}
\end{document}
