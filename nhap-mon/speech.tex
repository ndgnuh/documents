\documentclass[14pt]{extarticle}
\usepackage[a4paper,top=2cm,left=2cm,right=2cm,bottom=2cm]{geometry}
\usepackage[vietnamese]{babel}
\usepackage[utf8]{inputenc}
\usepackage{amsmath}
\usepackage{amsfonts}
\usepackage{amssymb}
\usepackage{amsthm}
\usepackage{tikz}
\usepackage{parskip}

\setlength{\parindent}{0pt}
\newcommand{\real}[1]{\operatorname{Re}{#1}}
\newcommand{\imag}[1]{\operatorname{Im}{#1}}

\begin{document}
\section{Tổng quan về ngành học}
Toán học là cơ sở cho bất kì lĩnh vực khoa học đương đại nào.
\textbf{Toán tin là gì}
Toán Tin, ngay ở cái tên của nó đã nói lên tất cả, đây là một ngành khoa học liên ngành giữa Toán và Tin học. Toán trong Toán Tin vì thế cũng đặc biệt quan tâm đến những nguyên lý cơ bản đằng sau thuật toán, cách thức xử lý thông tin, hoặc sử dụng máy tính, phần mềm như công cụ để giải quyết các bài toán tính toán cụ thể.

\textbf{Vai trò của toán tin trong thời đại này}
Toán học là một ngành khá đặc biệt, vì những ứng dụng của nó vừa có thể được sử dụng một một cách trực tiếp, vừa có thể sử dụng như cơ sở một số ngành khác. Toán thì không thiếu ứng dụng, nhưng với ngành học của chúng ta thì bọn em xin trình bày chủ yếu về "toán và tin".

Sự phát triển của máy móc, đặc biệt là máy tính đã làm thay đổi cuộc sống của con người. Nó cung cấp những phần mềm, công cụ hữu ích trong việc lao động, sản xuất, giao tiếp, giải trí, sinh hoạt,\ldots Vậy cái gì đã tạo nên những phần mềm đó. Có người nghĩ là những đoạn mã nguồn do các lập trình viên dùng những ngôn ngữ lập trình khác nhau viết ra. Nhưng suy cho cùng, ngôn ngữ lập trình, cũng như bao ngôn ngữ khác, chỉ đảm nhận vai trò \textit{giao tiếp} (ở đây là giữa người với máy).

Giả sử, ta bảo với một người sống ở 2000 năm trước rằng "bật cho tôi cái tv", dù có dùng ngôn ngữ họ hiểu được, thì họ cũng không thực hiện được vì bản thân người đó không có hiểu biết về vật tên là "cái tv", hay "bật tv" là cái gì. Theo lẽ đó, làm thế nào để một cái máy pha cà phê -- một vật vô tri -- có thể pha cà phê bằng một dòng lệnh như kiểu \texttt{Coffee.make(capuchino)}? Đây chính là nhiệm vụ của \textit{toán học} -- xử lý những logic đằng sau những dòng lệnh trên. 

Vậy có thể thấy rằng vai trò của toán học là vô cùng quan trọng, nó chính là linh hồn, của những chương trình. Trong thời đại hiện nay, với những công nghệ mới xuất hiện, những hệ quản trị tinh vi được cấu hình sẵn, đặc biệt hơn nữa, thứ được gọi là "máy học", việc xây dựng những chương trình trở nên đơn giản. Điều này khiến cho những chương trình phổ thông trở nên bão hòa, gần như ai cũng có thể tự xây dựng phần mềm của mình. Tuy nhiên, không phải ai cũng xây dựng được những phần mềm \textit{tốt}, vì điều này cần một thứ mà máy móc chưa làm được: ứng dụng toán học một cách linh hoạt. Do đó, vai trò của toán học trong thời đại ngày nay ngày càng to lớn. 

\textbf{Các định hướng của ngành Toán -- Tin}
\begin{itemize}
	\item Toán thuần túy: Toán lý thuyết nghiên cứu những lĩnh vực và giải quyết vấn đề mà tự thân nó đưa ra.
	\item Toán ứng dụng: Toán ứng dụng nghiên cứu cơ sở toán trong các ngành khoa học khác như Tin học, Vật lý, Y sinh,\ldots và ứng dụng của các lý thuyết này trong thế giới thực. bằng việc tìm ra vấn đề, mô hình hóa, giải quyết bài toán,\ldots
\end{itemize}

\textbf{So sánh Toán -- Tin với một số ngành khác}.

\textit{Khoa học máy tính} tìm cách giao tiếp với máy tính, cách tính toán khoa học,\ldots

\textit{Kĩ thuật máy tính} kết nối các thành phần của hệ thông máy tính với nhau, đồng bộ giữa các phần cứng và cả phần mềm chạy trên nó.

\textit{Công nghệ thông tin} là ngành khoa học nghiên cứu về việc ứng dụng công nghệ để xử lý thông tin.

\textit{Toán tin} có thể xem là hàm chứa những kiến thức cốt lõi của Khoa học máy tính và Kỹ thuật máy tính.

\textbf{Các kiến thức được trang bị}\par
Cung cấp tư duy logic, kiến thức nền tảng toán: giải tích, đại số tuyến tính, xác suất thống kê, tối ưu hóa, phương pháp tính, tối ưu tổ hợp, mô hình hóa, mô phỏng\ldots để mô hình hóa và giải quyết các bài toán thực tiễn như:
\begin{itemize}
	\item Phân tích dự báo
	\item Đánh giá kịch bản
	\item Quản trị rủi ro
	\item Tối ưu lợi nhuận
	\item \ldots
\end{itemize}

Cung cấp kiến thức nền tảng về tin học: cấu trúc dữ liệu, giải thuật, thuật toán, độ phức tạp, kĩ thuật lập trình, bảo mật, mạng máy tính\ldots

\textbf{Sinh viên toán tin ra trường làm gì?}
\begin{itemize}
	\item Các công ty, tập đoàn phần mềm
	\item Các viện nghiên cứu, các trường đại học
	\item Trung tâm Công nghệ thông tin, Phòng Tin học, Phòng Nghiên cứu khoa học, Phòng Thống kê, các đơn vị quản lý\ldots của:
		\begin{itemize}
			\item Các ngân hàng, công ty bảo hiểm, tập đoàn tài chính
			\item Các tập đoàn bưu chính viễn thông
			\item Các cơ quan hành chính nhà nước
			\item Các nhà máy, xí nghiệp, các doanh nghiệp\ldots
		\end{itemize}
	\item Bộ phận thống kê phân tích, dự báo, quản trị rủi ro, thẩm định đầu tư, định phí\ldots tại các ngân hàng, công ty tài chính, bảo hiểm, các doanh nghiệp\ldots
	\item Bộ phận nghiên cứu và ứng dụng toán trong giao thông, viễn thông, thủy lợi, nông nghiệp, công nghiệp, y tế\ldots
\end{itemize}

\section{Giả thuyết Riemann}

\textbf{Giả thuyết Riemann: }Những không điểm không tầm thường của hàm số $\zeta(s)$ đều có phần thực nằm trên đường thẳng $\real(s) = 1/2$.\par
Hàm zeta ($\zeta$):
\begin{equation*}
	\zeta(s) = \sum_{n=1}^\infty \frac 1 {n^s}
\end{equation*}
Hàm theta ($\theta$)
\begin{equation*}
	\theta(z,\tau) = \sum_{n=-\infty}^\infty \exp(\pi i n^2 \tau + 2\pi i n z)
\end{equation*}

\textbf{Lịch sử nghiên cứu:}\\
Thế kỉ 18, Euler đưa ra công thức trong một bài báo:
\begin{equation*}
	\zeta(s)=\sum_{n=1}^\infty \frac{1}{n^s} = \prod_p \frac{1}{1-p^{-s}}
\end{equation*}
trong đó $s$ là số thực, $p$ là tất cả các số nguyên tố. Euler đã chỉ ra rằng tích trên hội tụ với $s >1$. Đây là một phiên bản giải tích cho định lý cơ bản của số học.\par
Cuối thế kỉ 20, viện toán học Clay liệt bài toán này vào trong bảy bài toán thiên niên kỉ. Từ 1859 cho đến nay, qua  hơn 150 năm vẫn chưa có lời giải cho giả thuyết Riemann.\par
Bằng kiểm tra sức mạnh của máy tính người ta đã kiểm tra được có hàng tỉ không điểm trên đường thẳng $\real{s} = 1/2$\par
Năm 1914, nhà toán học Hardy đã chứng minh trên đường thẳng $\real(s) = 1/2$ có vô số không điểm của hàm $\zeta$
\begin{align*}
	\lim_{T \to \infty}N_0(T) = \infty
\end{align*}\par
Năm 1921, Hardy -- Little Wood:
\begin{align*}
	\label{eq:}
	N_0(T)	\ge C\cdot T, \quad C = \operatorname{const} > 0 
\end{align*}\par
Năm 1942, Albert Selberd:
\begin{align*}
	N_0(T) &\ge C\cdot T\log(T), \quad C = \operatorname{const}>0\\
	N_0(T) &\le T\log(T) \tag{Riemann-Malgolt}\\
	N_0(T) &\le CT\log(T)f(T), \quad f(T) \to \infty \textnormal{ khi }T \to \infty
\end{align*}\par
1974, Levinon:
\begin{align*}
	N_0(T) &> \frac 1 3 N(T)
\end{align*}\par
1989, Corney:
\begin{align*}
	N_0(T) > \frac 2 5 N(T)
\end{align*}\par
Trong đó $N(T)$ là số không điểm của $\zeta(s)$ mà có $0 <\imag{s} < T$, $N_0(T)$ là số không điểm mà $\real{s} = 1/2, 0 < \imag{s} < T$.

\textbf{Ý nghĩa}\par
Mật độ phân bố của số nguyên tố

\section{Toán trong thực tế}
\textbf{Operation Research}\par
\textit{Operation Research} (hay \textit{OR}) là một phương pháp phân tích dùng trong việc giải quyết vấn đề và đưa ra các quyết định. Trong OR, vấn đề được tách ra thành những phần cơ bản và xử lý trong các bước bằng giải tích toán học.

Những phương pháp toán học được sử dụng OR có thể bao gồm: logic toán học, mô phỏng, phân tích mạng lưới, lý thuyết hàng đợi, lý thuyết trò chơi\ldots

Việc thực hiện OR có thể tổng quát hóa thành ba bước sau:
\begin{enumerate}
	\item Xây dựng một tập hợp những cách giải quyết thô cho một vấn đề nào đó. Tập hợp này có thể lớn.
	\item Sau khi phân tích những cách giải quyết ở bước một được, đưa ra một tập hợp những giải pháp mà đã được chứng minh là khả thi.
	\item Những giải pháp lý thuyết thu được ở bước hai được xây dựng thành giải pháp trong thực tế, kiểm tra qua giả lập, (hoặc trong đời thực, nếu có thể)
\end{enumerate}

\textbf{Một số cách mô hình hóa bài toán thực tế}
\begin{itemize}
	\item Quy hoạch tuyến tính
	\item Quy hoạch lồi
	\item Quy hoạch hồi quy
	\item Quy hoạch động
	\item \ldots
\end{itemize}
\end{document}
