\documentclass[aspectratio=43]{beamer}
\usepackage[vietnamese]{babel}
\usepackage[utf8]{inputenc}
\usepackage{amsmath}
\usepackage{amsfonts}
\usepackage{amssymb}
\usepackage{amsthm}
\usepackage{tikz}
% \usepackage[a4paper]{geometry}
\usetheme{material}

\usetikzlibrary{shapes.geometric, arrows}
\tikzstyle{process}=[rectangle, align=center, minimum width=3cm, minimum height=1cm, draw=black, text centered]
\tikzstyle{startstop}=[rectangle, align=center, rounded corners, minimum width=3cm, minimum height=1cm, draw=black, text centered]
\tikzstyle{io}=[trapezium, align=center, trapezium left angle=60, minimum width=3cm, minimum height=1cm, draw=black, text centered]
\tikzstyle{decision}=[diamond, aspect=2, align=center, minimum width=3cm, minimum height=1cm, draw=black, text centered]
\tikzstyle{arrow} = [thick,->,>=stealth]

\usefonttheme{serif}
\title{Báo cáo nhập môn toán tin}
\date{Ngày 20 tháng 05 năm 2019}
\author{
	Đỗ Ngọc Dũng -- 2017\\
	Nguyễn Đức Hùng -- 20173520\\
	Phùng Anh Hùng -- 2017\\
	Ngô Việt Trung -- 2017
}
\begin{document}
\begin{frame}
	\maketitle
\end{frame}

\begin{frame}{Mục lục}
\tableofcontents	
\end{frame}

\section{Tổng quan về ngành học}
\begin{frame}{\secname}	
	\begin{itemize}
		\item Toán tin là gì?
		\item Toán tin có vai trò như nào trong thời đại này?
		\item Các định hướng của ngành?
		\item Sinh viên toán tin ra trường làm những gì?
	\end{itemize}
\end{frame}

\section{Kĩ năng phát triển bản thân}
\begin{frame}{\secname}
	\begin{itemize}
		\item Các chỉ số IEAQ
		\item Mô hình phát triển con người ASK
		\item Phương pháp học LIPE
		\item Nguyên tắc đặt mục tiêu SMART
		\item Tháp nhận thức Bloom
		\item Tháp nhu cầu Maslow
	\end{itemize}	
\end{frame}

\section{Toán ứng dụng trong thực tế}
\begin{frame}{\secname}
	\begin{itemize}
		\item Tối ưu đa mục tiêu
		\item Operation research
		\item Mô hình hóa bài toán
	\end{itemize}	
\end{frame}

\begin{frame}{\secname}
	\begin{figure}[htpb]
	\begin{center}
		\begin{tikzpicture}[scale=0.8, node distance=1.5cm, transform shape]
			\node (s0) [startstop]  {Bài toán thực tế};	
			\node (s1) [below of=s0, process]  {Mô hình hóa \& xử lý};	
			\node (s2) [below of=s1, process]  {Code};	
			\node (s3) [below of=s2, decision, yshift=-0.5cm]  {Giám sát\\kiểm thử};	
			\node (s4) [below of=s3, startstop, yshift=-0.5cm]  {Kết thúc quá trình};	
			\node (s5) [right of=s3, process, xshift=5cm] {Tinh chỉnh};
			\draw[arrow] (s0) -> (s1);
			\draw[arrow] (s1) -> (s2);
			\draw[arrow] (s2) -> (s3);
			\draw[arrow] (s3) ->node[anchor=west] {ok} (s4);
			\draw[arrow] (s3) ->node[anchor=south] {Không ok} (s5);
			\draw[arrow] (s5) |- (s1);
		\end{tikzpicture}
	\end{center}
	\caption{Giải quyết bài toán trong thực tế}
	\label{fig:giải_quyết_bài_toán_trong_thực_tế}
\end{figure}

\end{frame}

\section{Giả thuyết Riemann}
\begin{frame}{\secname}
	\begin{itemize}
		\item Lịch sử nghiên cứu
		\item Hàm $\zeta$ và phương trình hàm $\zeta$
	\end{itemize}	
\end{frame}
\end{document}
