\begin{definition}
	[Chuỗi Laurent]	
	\label{def:laurent}
	Khai triển Laurent của hàm \(f(z)\) quanh điểm \(z_0\) là chuỗi lũy thừa:
	\begin{equation}
		\label{Laurent}
		f(z) = \sum_{n = -\infty}^\infty a_n (z-z_0)^n
	\end{equation}
	Trong đó \(a_n\) được xác định bởi:
	\begin{equation}
		\label{LaurentTermCoef}
		a_n = \frac{1}{2\pi i}\oint_\gamma \frac{f(z)}{(z-z_0)^{n+1}}dz
	\end{equation}
\end{definition}


Với \(\gamma\) là đường cong kín bao quanh \(z_0\) sao cho: tồn tại đường tròn \(\beta\) tâm \(z_0\), bán kính \(r > 0\), \(f(z)\) giải tích trên miền được giới hạn bởi \(\beta \cap \gamma\). 
\begin{figure}[h]
	\centering
	% \documentclass{article}
% \usepackage{tikz}
% \usepackage{xcolor}
% \definecolor{pathfill}{HTML}{a3be8c}
\begin{tikzpicture}
  \path[fill=pathfill, draw=pathfill, ultra thick]
    (3.75,1.75) -- (3.5,-2) -- (-1.5,-3)  -- (-3,-0.5)  ;
  \path[fill=pathfill,draw=black] 
    (3.75,1.75) to [in=60, out=120] 
    (-1.5,2.75) to[in=60, out=240] (-3,-0.5)
    (-3,-0.5)  to[in=180, out=-135] (-1.5,-3)
    (-1.5,-3) to[in=-120, out=15] (3.5,-2)
    (3.5,-2) to[in=-45, out=60] (3.75,1.75);
  \path[fill=white,draw=black]
    (2,0) arc[radius=2, start angle=0, end angle=360] (0,0);
  \draw[-latex']  (3.74,1.74) to[in=-45, out=60] (3.75,1.75);
  \node at (0,0) {\textbullet};
  \node[anchor=south west] at (0,0) {\(z_0\)};
  \node[anchor=south east] at (2,0) {\(\beta\)};
  \node[anchor=south west] at (3.75,1.75) {\(\gamma\)};
\end{tikzpicture}
% \end{document}
	\caption{Đường cong $\gamma$}
	\label{fig.1}
\end{figure}

Đặc biệt, khi \(n = -1\), ta có:
\begin{equation}
	\label{a-1}
	a_{(-1)} = \frac{1}{2\pi i }\oint_\gamma f(z)dz
\end{equation}

\begin{definition}
	\label{def:residue}
	\(a_{-1}\) được gọi là thặng dư (residue) của \(f(z)\) tại điểm \(z_0\). Ký hiệu:
	\begin{equation}
		\label{eq:residue}
		\Res{f(z)}{z=z_0}= \frac{1}{2\pi i }\oint_\gamma f(z)dz
	\end{equation}
\end{definition}

Với một hàm \(f(z)\) bất kì giải tích trên miền \(D\setminus \{z_0\}\), \(z_0\) là cực điểm cấp \(n\) của \(f(z)\), ta có thể tìm thặng dư của \(f(z)\) tại \(z_0\) bằng cách:
\begin{equation}
	\label{find-res}
	\Res{f(z)}{z=z_0} = \left. \frac{1}{(n-1)!} \cdot\frac{d^{n-1}}{dz^{n-1}}\left[(z-z_0)^nf(z)\right] \right\rvert_{z=z_0}
\end{equation}

Trong trường hợp \(z_0\) là cực điểm đơn giản:
\begin{equation}
	\label{find-res-simple-pole}
	\Res{f(z)}{z=z_0} = \left. \frac{p(z)}{q'(z)} \right\rvert_{z=z_0} \text{ với } f(z) = \frac{p(z)}{q(z)}
\end{equation}

\begin{theorem}
	\label{theorem:residue_integral}
	Nếu hàm $f(z)$ có các cực điểm $z_k$ trên miền $D$ giới hạn bởi đường cong $C$, ta có:
	\begin{equation*}
		\oint_{C} f(z)dz = 2\pi i \sum_{\forall z_k} \Res{f(z)}{z=z_k}
	\end{equation*}
\end{theorem}
